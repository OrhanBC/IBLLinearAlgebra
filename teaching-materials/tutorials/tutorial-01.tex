\documentclass[red]{tutorial}
\usepackage[no-math]{fontspec}
\usepackage{xpatch}
	\renewcommand{\ttdefault}{ul9}
	\xpatchcmd{\ttfamily}{\selectfont}{\fontencoding{T1}\selectfont}{}{}
	\DeclareTextCommand{\nobreakspace}{T1}{\leavevmode\nobreak\ }
\usepackage{polyglossia} % English please
	\setdefaultlanguage[variant=us]{english}
%\usepackage[charter,cal=cmcal]{mathdesign} %different font
%\usepackage{avant}
\usepackage{microtype} % Less badboxes


\usepackage[charter,cal=cmcal]{mathdesign} %different font
%\usepackage{euler}
 
\usepackage{blindtext}
\usepackage{calc, ifthen, xparse, xspace}
\usepackage{makeidx}
\usepackage[hidelinks, urlcolor=blue]{hyperref}   % Internal hyperlinks
\usepackage{mathtools} % replaces amsmath
\usepackage{bbm} %lower case blackboard font
\usepackage{amsthm, bm}
\usepackage{thmtools} % be able to repeat a theorem
\usepackage{thm-restate}
\usepackage{graphicx}
\usepackage{xcolor}
\usepackage{multicol}
\usepackage{fnpct} % fancy footnote spacing

 
\newcommand{\xh}{{{\mathbf e}_1}}
\newcommand{\yh}{{{\mathbf e}_2}}
\newcommand{\zh}{{{\mathbf e}_3}}
\newcommand{\R}{\mathbb{R}}
\newcommand{\Z}{\mathbb{Z}}
\newcommand{\N}{\mathbb{N}}
\newcommand{\proj}{\mathrm{proj}}
\newcommand{\Proj}{\mathrm{proj}}
\newcommand{\Perp}{\mathrm{perp}}
\renewcommand{\span}{\mathrm{span}\,}
\newcommand{\Span}{\mathrm{span}\,}
\newcommand{\Img}{\mathrm{img}\,}
\newcommand{\Null}{\mathrm{null}\,}
\newcommand{\Range}{\mathrm{range}\,}
\newcommand{\rref}{\mathrm{rref}}
\newcommand{\rank}{\mathrm{rank}}
\newcommand{\Rank}{\mathrm{rank}}
\newcommand{\nnul}{\mathrm{nullity}}
\newcommand{\mat}[1]{\begin{bmatrix}#1\end{bmatrix}}
\newcommand{\chr}{\mathrm{char}}
\renewcommand{\d}{\mathrm{d}}


\theoremstyle{definition}
\newtheorem{example}{Example}[section]
\newtheorem{defn}{Definition}[section]

\theoremstyle{theorem}
\newtheorem{thm}{Theorem}[section]

\pgfkeys{/tutorial,
	name={Tutorial 1},
	author={Jason Siefken},
	course={MAT 223},
	date={},
	term={},
	title={Math \& the World}
	}

\begin{document}
	\begin{tutorial}
				\begin{objectives}
			In this tutorial you will work on rephrasing problems
			with mathematical language---this is an essential skill if you ever
			plan on applying mathematical techniques to the world!

			These problems relate to the following course learning
			objective: \textit{work independently to understand concepts and procedures that have not
			been previously explained to you}.
		\end{objectives}

	%	\bigskip


		\subsection*{Problems}


		\begin{enumerate}
			\item Use vectors, sets, set operations, and set-builder notation
				to describe the following as subsets of $\R^2$.
				\begin{enumerate}
					\item The $x$-axis.
					\item The corners of a square $S$, which is centered at the origin and whose
						sides have length 3 and are aligned with the axes.
					\item The diagonal of $S$ (from before) starting from the lower-left to the upper-right.
					\item Both diagonals of $S$.
					\item The line segment from $(2,3)$ to $(4,1)$, including the endpoints.
					\item The line segment from $(2,3)$ to $(4,1)$, not including the endpoints.
				\end{enumerate}

			\item Let's make a smiley face!\footnote{ This question is not a joke, and a version of
				it may show up on your midterm.}
				\begin{enumerate}
					\item Describe the lower half of a circle of radius 1 centered at the origin.
						Call this set $M$ (for mouth!).
					\item Pick a location for the eyes and describe them as small, filled in
						circles. Call the left eye $L$ and the right eye $R$.
					\item Describe the whole face using $M$, $L$, and $R$. Call the face $F$.
					\item When we draw a set, we usually draw black
						for points in the set and leave points not in the set white.
						Let's draw a reverse-face. Come up with a set $F_R$
						for a face where the ``skin'' of the face is included in $F_R$, but
						the eyes and mouth are not.
				\end{enumerate}
			\item \emph{Interpolation} is the process of filling in points that might not exist already.
				It's commonly used when zooming-in or rotating a picture on your computer.  The picture $P$ consists
				of four colored pixels, \emph{\color{red}red} at $(0,0)$, \emph{\color{green!70!black}green} at $(1,0)$,
				and \emph{\color{blue}blue} at $(2,0)$ and $(3,0)$. To your brain, what is important about this picture
				is the \emph{relative spacing} between the colors, not their absolute positions. We will interpolate the color
				positions for a transformed $P$.
				\begin{enumerate}
					\item Give the coordinates of each color if $P$ were translated, going from $(1,4)$ to $(4,4)$.
					\item Give the coordinates of each color if $P$ were twice as big, going from $(0,0)$ to $(6,0)$.
					\item Give the coordinates of each color if $P$ were rotated and zoomed, going from $(-1,-1)$ to $(7,0)$.
				\end{enumerate}

		\end{enumerate}

	\end{tutorial}


	\begin{solutions}
		\input{../../book/tutorials/tutorial1-solutions.tex}
	
	\end{solutions}
	\begin{instructions}
		\input{../../book/tutorials/tutorial1-instructions.tex}
	\end{instructions}

\end{document}
