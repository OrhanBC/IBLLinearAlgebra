\begin{exercises}
	\begin{problist}
		\prob 
		Let
		\[
			\mathcal A=\Set*{\mat{2\\1}_{\mathcal E},
			\mat{1\\-2}_{\mathcal E}}\quad\text{and}\quad
			\mathcal B=\Set*{\mat{3\\-1}_{\mathcal E},
			\mat{-2\\3}_{\mathcal E}}.
		\]
		be bases for $\R^2$. Define $\vec x\in \R^2$ by $[\vec x]_{\mathcal A}=\mat{1\\-1}$.
		\begin{enumerate}
			\item	Find $[\vec x]_{\mathcal E}$ and
				 $[\vec x]_{\mathcal B}$.

			\item Find the change of basis matrices 
				$\BasisChange{\mathcal A}{\mathcal E}$,
				$\BasisChange{\mathcal E}{\mathcal A}$,
				$\BasisChange{\mathcal A}{\mathcal B}$, and
				$\BasisChange{\mathcal B}{\mathcal A}$.
		\end{enumerate}
		\begin{solution}
			\begin{enumerate}
				\item Note that by definition
				\[
					A=\Set{2\vec e_1+\vec e_2, \vec e_1-2\vec e_2}.
				\]
				
				Since $[\vec x]_{\mathcal A}=\mat{1\\-1}$, it follows that
				\[
					\vec x=(2\vec e_1+\vec e_2)-(\vec e_1-2\vec e_2)=\vec e_1+3\vec e_2.
				\]
				
				It thus follows that $[\vec x]_{\mathcal E}=\mat{1\\3}$.
				
				To find $[\vec x]_{\mathcal B}$, we must first express the two elements of
				$\mathcal A$ as linear combinations of the two elements of $\mathcal B$.
				This involves solving two systems of linear equations:
				\[
					2\vec e_1+\vec e_2=x_1(3\vec e_1-\vec e_2)+x_2(-2\vec e_1+3\vec e_2)
				\]
				and
				\[
					\vec e_1-2\vec e_2=y_1(3\vec e_1-\vec e_2)+y_2(-2\vec e_1+3\vec e_2).
				\]
				
				Written as column vectors, these are
				\[
					\mat{2\\1}=x_1\mat{3\\-1}+x_2\mat{-2\\3}
				\]
				and
				\[
					\mat{1\\-2}=y_1\mat{3\\-1}+y_2\mat{-2\\3}.
				\]
				
				Solving the systems and applying the definition, we obtain
				\[
					[\vec a_1]_{\mathcal B}=\mat{x_1\\x_2}=\frac{1}{7}\mat{8\\5},\quad
					[\vec a_2]_{\mathcal B}=\mat{y_1\\y_2}=\frac{1}{7}\mat{-1\\-5}.
				\]
				
				Now we have
				\[
					[\vec x]_{\mathcal B}=[\vec a_1]_{\mathcal B}-[\vec a_2]_{\mathcal B}=\frac{1}{7}\mat{8\\5}-\frac{1}{7}\mat{8\\5}=\frac{1}{7}\mat{9\\10}.
				\]
				\item By definition, $\BasisChange{\mathcal A}{\mathcal E}$ is the matrix $M$ satisfying
				\[
					M[\vec x]_{\mathcal A}=[\vec x]_{\mathcal E}
				\]
				for all vectors $\vec x$. In particular, it must satisfy
				\[
					M[\vec a_1]_{\mathcal A}=[\vec a_1]_{\mathcal E},\quad
					M[\vec a_2]_{\mathcal A}=[\vec a_2]_{\mathcal E}.
				\]
				
				It thus follows that $M$ satisfies
				\[
					M\mat{1\\0}=\mat{2\\1},\quad
					M\mat{0\\1}=\mat{1\\-2}.
				\]
				
				Solving for $M$ gives
				\[
					M=
					\begin{bmatrix}
						2 & 1 \\ 1 & -2
					\end{bmatrix}.
				\]
				
				To find $\BasisChange{\mathcal E}{\mathcal A}$, we simply need to compute $M^{-1}$.
				Following the explicit inverse formula for $2\times2$ matrices, we see that
				\[
					\BasisChange{\mathcal E}{\mathcal A}=M^{-1}=\frac{1}{5}
					\begin{bmatrix}
						2 & 1 \\ 1 & -2
					\end{bmatrix}.
				\]
				
				To compute $\BasisChange{\mathcal A}{\mathcal B}$, we note that
				\[
					\BasisChange{\mathcal A}{\mathcal B}=\BasisChange{\mathcal E}{\mathcal B}\BasisChange{\mathcal A}{\mathcal E}.
				\]
				
				We compute $\BasisChange{\mathcal E}{\mathcal B}$ readily as
				\[
					\BasisChange{\mathcal B}{\mathcal E}=
					\begin{bmatrix}
						3 & -2 \\ -1 & 3
					\end{bmatrix}.
				\]
				
				Thus,
				\[
					\BasisChange{\mathcal E}{\mathcal B}=\BasisChange{\mathcal B}{\mathcal E}^{-1}=\frac{1}{7}
					\begin{bmatrix}
						3 & 2 \\ 1 & 3
					\end{bmatrix}.
				\]
				
				Therefore,
				\[
					\BasisChange{\mathcal A}{\mathcal B}=\frac{1}{7}
					\begin{bmatrix}
						3 & 2 \\ 1 & 3
					\end{bmatrix}
					\begin{bmatrix}
						2 & 1 \\ 1 & -2
					\end{bmatrix}=
					\frac{1}{7}
					\begin{bmatrix}
						8 & -1 \\ 5 & -5
					\end{bmatrix}.
				\]
				
				Taking inverse gives $\BasisChange{\mathcal B}{\mathcal A}$.
			\end{enumerate}
		\end{solution}
		\prob 
		Let
		\[
			\mathcal A=\Set*{
				\mat{2\\1\\0}_{\mathcal E},
				\mat{1\\-2\\0}_{\mathcal E},
				\mat{0\\0\\1}_{\mathcal E}
			}\quad\text{and}\quad
			\mathcal B=\Set*{\vec b_1,\vec b_2,\vec b_3
			}
		\]
		be bases for $\R^3$ where
		\[
			\vec b_1=\mat{1\\0\\0}_{\mathcal A}
			\qquad
			\vec b_2=\mat{1\\1\\0}_{\mathcal A}
			\qquad
			\vec b_3=\mat{1\\1\\1}_{\mathcal A}.
		\]
		\begin{enumerate}
			\item Find the representation of $\vec b_1$, $\vec b_2$, and $\vec b_3$
				in the standard basis.
			\item Find the change of basis matrices $\BasisChange{\mathcal E}{\mathcal A}$
				and $\BasisChange{\mathcal B}{\mathcal E}$.
			\item Use $\BasisChange{\mathcal E}{\mathcal A}$
				and $\BasisChange{\mathcal B}{\mathcal E}$ to compute
				$\BasisChange{\mathcal B}{\mathcal A}$.
		\end{enumerate}
		\begin{solution}
			\begin{enumerate}
				\item By definition, we have
				\[
					\begin{aligned}
						[\vec b_1]_{\mathcal E}&=[\vec a_1]_{\mathcal E}=2\vec e_1+\vec e_2,\\
						[\vec b_2]_{\mathcal E}&=[\vec a_1]_{\mathcal E}+[\vec a_2]_{\mathcal E}\\
						&=(2\vec e_1+\vec e_2)+(\vec e_1-2\vec e_2)=3\vec e_1-\vec e_2,\\
						[\vec b_3]_{\mathcal E}&=[\vec a_1]_{\mathcal E}+[\vec a_2]_{\mathcal E}+[\vec a_3]_{\mathcal E}\\
						&=(2\vec e_1+\vec e_2)+(\vec e_1-2\vec e_2)+(\vec e_3)\\
						&=3\vec e_1-\vec e_2+\vec e_3.
					\end{aligned}
				\]
				\item We have that
				\[
					\BasisChange{\mathcal A}{\mathcal E}=
					\begin{bmatrix}
						2 & 1 & 0\\
						1 & -2 & 0\\
						0 & 0 & 1
					\end{bmatrix}.
				\]
				
				By part (a), we have
				\[
					\BasisChange{\mathcal B}{\mathcal E}=
					\begin{bmatrix}
						2 & 3 & 3\\
						1 & -1 & -1\\
						0 & 0 & 1
					\end{bmatrix}.
				\]
				
				Inverting the former, we see that
				\[
					\BasisChange{\mathcal E}{\mathcal A}=\frac{1}{5}
					\begin{bmatrix}
						2 & 1 & 0\\
						1 & -2 & 0\\
						0 & 0 & 5
					\end{bmatrix}.
				\]
				\item We have
				\[
					\begin{aligned}
						\BasisChange{\mathcal B}{\mathcal A}
						&=\BasisChange{\mathcal E}{\mathcal A}\BasisChange{\mathcal B}{\mathcal E}\\
						&=\frac{1}{5}
						\begin{bmatrix}
							2 & 1 & 0\\
							1 & -2 & 0\\
							0 & 0 & 5
						\end{bmatrix}
						\begin{bmatrix}
							2 & 3 & 3\\
							1 & -1 & -1\\
							0 & 0 & 1
						\end{bmatrix}\\
						&=
						\begin{bmatrix}
							1 & 1 & 1\\
							0 & 1 & 1\\
							0 & 0 & 1
						\end{bmatrix}.
					\end{aligned}
				\]
			\end{enumerate}
		\end{solution}
		\prob Let $\mathcal B=\Set*{\mat{1\\0}_{\mathcal E},\mat{1\\1}_{\mathcal E}}$.
		For each linear transformation
		$\mathcal T:\R^{2}\to\R^{2}$ defined below, compute
		$[\mathcal T]_{\mathcal E}$ and $[\mathcal T]_{\mathcal B}$.
		\begin{enumerate}
			\item Let $\mathcal T$ be the transformation that rotates
				every vector counter clockwise by $90^{\circ}$.

			\item Let $\mathcal T$ be the transformation that projects
				every vector onto the $y$-axis.
			
			\item Let $\mathcal T$ be the transformation that doubles
				every vector.

			\item Let $\mathcal T$ be the transformation that reflects
				every vector over the line $y=x$.
		\end{enumerate}
		\begin{solution}
			\begin{enumerate}
				\item 
					$[\mathcal T]_{\mathcal E}=\mat{0&-1\\1&0}$ and $[\mathcal T]_{\mathcal B}=
				\begin{bmatrix}
				-1 & -2 \\ 1 & 1
				\end{bmatrix}$
				\item 
				$[\mathcal T]_{\mathcal E}=
				\begin{bmatrix}
					0 & 0 \\ 0 & 1
				\end{bmatrix}$
				
				To compute $[\mathcal T]_{\mathcal B}$, note that $\mathcal T\vec b_1=\mathcal T\vec e_1=\vec 0$
				and $\mathcal T\vec b_2=\mathcal T(\vec e_1+\vec e_2)=\vec e_2$. Further,
				$\vec e_2=-\vec b_1+\vec b_2$. Therefore, 
				$[\mathcal T]_{\mathcal B}=
				\begin{bmatrix}
					0 & -1 \\ 0 & 1
				\end{bmatrix}$.
				\item 
				$[\mathcal T]_{\mathcal E}=[\mathcal T]_{\mathcal B}=
				\begin{bmatrix}
					2 & 0 \\ 0 & 2
				\end{bmatrix}$
				\item Note that $\mathcal T\vec e_1=\vec e_2$ and $\mathcal T\vec e_2=\vec e_1$, so
				$[\mathcal T]_{\mathcal E}=
				\begin{bmatrix}
					0 & 1 \\ 1 & 0
				\end{bmatrix}$.
				To compute $[\mathcal T]_{\mathcal B}$, we simply observe that $\vec b_1=\vec e_1$
				and $\vec b_2=\vec e_1+\vec e_2$. Thus, $\mathcal T\vec b_1=\vec e_2=-\vec b_1+\vec b_2$
				and $\mathcal T\vec b_2=\vec e_1+\vec e_2=\vec b_2$. Therefore, 
				$[\mathcal T]_{\mathcal B}=
				\begin{bmatrix}
					-1 & 0 \\ 1 & 1
				\end{bmatrix}$.
			\end{enumerate}
		\end{solution}
		\prob For each statement below, determine whether it is true or false. Justify your answer.
		\begin{enumerate}
			\item Any invertible $n \times n$ matrix can be viewed as
				a change of basis matrix.

			\item Any $n \times n$ matrix is similar to itself.

			\item Let $A$ be an $m \times n$ matrix. If $m
				\neq n$, then there is not matrix that is
				similar to $A$.

			\item Any invertible $n \times n$ matrix $A$ is similar to
				$A^{-1}$ since $AA^{-1}=I$.
		\end{enumerate}
		\begin{solution}
			\begin{enumerate}
				\item True. A square matrix $M$ is invertible if and only if it is a change of basis matrix.
				\item True. Any square matrix $M$ satsifies $M=IMI^{-1}$.
				\item True. The definition of similarity requires the existence of an invertible matrix $P$,
					i.e. a square matrix, and a matrix $B$ such that $B=PAP^{-1}$. If $A$ is not a square matrix,
					then either $PA$ or $AP^{-1}$ is not defined, so such a matrix $P$ cannot exist.
				\item False. For example, 
				$A=\begin{bmatrix}
					2 & 0 \\ 0 & -1
				\end{bmatrix}$
				is not similar to its inverse.
			\end{enumerate}
		\end{solution}
	\end{problist}
\end{exercises}
