\begin{exercises}
	\begin{problist}
		\prob For each equation given below, determine if it is a linear
		equation. If not, explain what makes it nonlinear.
		\begin{enumerate}
			\item $\cos(4)x_{1}+\mathrm{e}y_{2}+\pi z_{3}=\mathrm{e}^{\pi}$

			\item $4x_{1}+2x_{2}+5x_{4}=4x_{2}+4x_{5}+5$

			\item $5x+2y+8z=\cos(y)$

			\item $12x+3xy+5z=2$

			\item $\cos(4)x+\sin(4)y=\tan(4)x$

			\item $\frac{x}{y}=1$
		\end{enumerate}
		\begin{solution}
			\begin{enumerate}
				\item Linear equation.

				\item Linear equation.

				\item Not a linear equation because of the $\cos(y)$ term.

				\item Not a linear equation because of the $3xy$ term.

				\item Linear equation.

				\item Not a linear equation because of the $\frac{x}{y}$ term.
					Note that it is \emph{almost} equivalent to the equation $x=y$,
					but they are not equivalent because $x = 0,y = 0$ is a
					solution to the latter equation but not the former.
			\end{enumerate}
		\end{solution}

		\prob Convert each vector equation given below to a system of linear
		equations.

		\begin{enumerate}
			\item $x\mat{1 \\ -1 \\ 0}+y\mat{0 \\ 1 \\ 0}+z\mat{4 \\ 6 \\ 1}=\mat
				{2 \\ -5 \\ 2}$

			\item $x\mat{7 \\ 16}+y\mat{8 \\ 13}=\mat{11 \\ 30}$

			\item $\vec u+t\vec u - s(\vec v+\vec w)=\vec 0$ where $\vec u=\mat{1\\1}$,
				$\vec v=\mat{2\\-1}$, and $\vec w=\mat{3\\4}$.
		\end{enumerate}
		\begin{solution}
			\begin{enumerate}
				\item $\systeme{x+4z=2, -x+y+6z=-5, z=2}$

				\item $\systeme{7x+8y=11, 16x+13y=30}$

				\item $\systeme{t-5s=-1, t-3s=-1}$
			\end{enumerate}
		\end{solution}

		\prob Convert each system of linear equations given below to a vector equation.
		\begin{enumerate}
			\item $\systeme{4x_2+2x_3=0, x_1+2x_3=0, 9x_2+2x_3=1}$

			\item $\systeme{0x+0y+0z=0, x+y+z=3}$
		\end{enumerate}

		\begin{solution}
			\begin{enumerate}
				\item $x_{1}\mat{0 \\ 1 \\ 0}+x_{2}\mat{4 \\ 0 \\ 9}+x_{3}\mat{2 \\ 2 \\ 2}
					=\mat{0 \\ 0 \\ 1}$

				\item $x\mat{0 \\ 1}+y\mat{0 \\ 1}+z\mat{0 \\ 1}=\mat
					{0 \\ 3}$
			\end{enumerate}
		\end{solution}

		\prob Consider the vector equation $x\mat{2 \\ 4}+y\mat{8 \\ 16}=\vec b$
		where $\vec b$ is unknown.
		\begin{enumerate}
			\item Show that if $\vec b=\mat{7 \\ 14}$, the system is consistent.

			\item Are there other vectors $\vec b$ that make the system consistent?
				If so, how many? Justify your answer.

			\item Show that if $\vec b=\mat{5 \\ 12}$, the system is
				inconsistent.

			\item Are there other vectors $\vec b$ that make the system inconsistent?
				If so, how many? Justify your answer.
		\end{enumerate}

		\begin{solution}
			\begin{enumerate}
				\item If $\vec b=\mat{7 \\ 14}$, then the vector equation
					becomes
					\[
						x\mat{2 \\ 4}+y\mat{8 \\ 16}=\mat{7 \\ 14}.
					\]
					Converting it to a system of linear equations and row
					reducing we get
					\[
						\systeme{2x+8y=7,4x+16y=14}\rightarrow \systeme{x+4y=3.5, 0x+0y=0}.
					\]
					The solution to this system is then
					\[
						\left\{
						\begin{array}
							{ccc} x & = & 3.5-4t \\ y & = & t
						\end{array}\right. (t\in \R).
					\]
					This system is consistent.

				\item There are vectors $\vec{b}$ that makes the system consistent.
					For instance, any vector $\vec b = \vec b=\mat{t \\ 2t}$ where
					$t\in\R$ makes the system consistent. Since there
					are infinitely many real numbers, we conclude that there are
					infinitely many vectors $\vec b$ that makes the system consistent.

				\item If $\vec b=\mat{5 \\ 14}$, then the vector equation
					becomes
					\[
						x\mat{2 \\ 4}+y\mat{8 \\ 16}=\mat{5 \\ 12}.
					\]
					Converting it to a system of linear equations and row
					reducing we get
					\[
						\systeme{2x+8y=5,4x+16y=12}\rightarrow \systeme{x+4y=2.5, 0x+0y=2}.
					\]
					This system is inconsistent.

				\item There are vectors $\vec b$ that makes the system inconsistent.
					For instance, $\mat{10 \\ 24}$ is is such a vector. In
					general, any vector $\vec b$ with $\vec b=\mat{5t \\ 12t}$ where
					$t\in\R$ ($t\ne 0$) makes the system inconsistent.
					Since there are infinitely many real numbers, we conclude that
					there are infinitely many vectors $\vec b$ that makes the system
					inconsistent.
			\end{enumerate}
		\end{solution}

		\prob On Kokoro's farm, there is a cage with $35$ animals, some of which
		are chickens and some of which are rabbits. Kokoro counted the total
		number of legs in the cage and found that there were $94$ legs in all (notably,
		each chicken has exactly two legs and each rabbit has four legs). Kokoro
		decides to use this information to figure out how many chickens and how
		many rabbits there are\footnote{ This problem based on a classical
		Chinese problem from the ancient Chinese treatise \emph{Mathematical
		Classic of Master Sun} (or \emph{Sunzi Suanjing}) written during 3rd to
		5th centuries \textsc{a.d.}}.

		\begin{enumerate}
			\item Set up a system of linear equations that you could solve to answer
				Kokoro's question.

			\item Is the system consistent? If so, answer Kokoro's question.

			\item Kokoro wants to set up three other cages. For each described
				cage below, explain using complete English sentences, whether
				such a configuration is possible. Justify your answers using linear
				algebra.
				\begin{enumerate}
					\item Kokoro wants to set up a cage with \emph{cats} and
						\emph{dogs} (notably, each cat has exactly four legs and
						each dog has four legs) so that there are $35$ animals
						in total, and the total number of legs is $94$.

					\item Kokoro wants to set up a cage with \emph{cats} and
						\emph{dogs} so that there are $35$ animals in total, and
						the total number of legs is $140$.

					\item Kokoro wants to set up a cage with \emph{chickens} and
						\emph{rabbits} so that there are $42$ animals in total,
						and the total number of legs is $77$.
				\end{enumerate}
		\end{enumerate}
		\begin{solution}
			\begin{enumerate}
				\item Let $x$ be the number of chickens, and let $y$ be the
					number of rabbits. Using the information given in the problem,
					we have
					\[
						\systeme{x+y=35, 2x+4y=94}.
					\]

				\item Row reducing
					\[
						\systeme{x+y=35, 2x+4y=94},
					\]
					we get
					\[
						\systeme{x+y=35, y=12}.
					\]
					This shows that the system is consistent. The solution to
					this system is $x=23, y=12$. Thus, there are 23 chickens and
					12 rabbits in the farm.

				\item Before discussing each configuration, we point out that a
					configuration is possible if there exists a natural number
					solution to the system of linear equations associated with
					the configuration.
					\begin{enumerate}
						\item For the first configuration, let $x$ be the number
							of cats, and let $y$ be the number of dogs. Using
							the information given in the problem, we have
							\[
								\systeme{x+y=35, 4x+4y=94}.
							\]
							Row reducing this system, we get
							\[
								\systeme{x+y=35, 0x+0y=-46}.
							\]
							This system is inconsistent, which means there's no solution
							to this system. Therefore, Kokoro's first configuration
							is not possible.

						\item For the second configuration, let $x$ be the
							number of cats, and let $y$ be the number of dogs.
							Using the information given in the problem, we have
							\[
								\systeme{x+y=35, 4x+4y=140}.
							\]
							Row reducing this system, we get
							\[
								\systeme{x+y=35, 0x+0y=0}.
							\]
							This system is consistent, and the complete solution
							is given by
							\[
								\systeme{x=35-t, y=t}(t\in \R).
							\]
							Take $t=1$, and we get a natural number solution
							$x=34,y=1$. (In fact, there is more than one natural
							number solution.) Therefore, Kokoro's second
							configuration is possible.

						\item For the third configuration, let $x$ be the number
							of chickens, and let $y$ be the number of rabbits.
							Using the information given in the problem, we have
							\[
								\systeme{x+y=42, 2x+4y=77}.
							\]
							Row reducing this system, we get
							\[
								\systeme{x+y=42, y=-\frac{7}{2}}.
							\]
							This system is consistent and the unique solution is
							$x = \frac{91}{2}, y = -\frac{7}{2}$ However, there cannot
							be 91/2 of a chicken, so Kokoro's third
							configuration is not possible.
					\end{enumerate}
			\end{enumerate}
		\end{solution}

		\prob For each statement below, determine whether it is true or false.
		Justify your answer.
		\begin{enumerate}
			\item A system of linear equations of 4 variables with 3 equations is
				always consistent.

			\item Any system of linear equation with $0x_{1}+0x_{2}+\cdots+0x_{n}
				=0$ being one of the equations must be consistent.

			\item There are $m,c\in \R$ so that the $y$-axis is the solution set
				to the equation $y=mx+c$.

			\item There are $m,c\in \R$ so that the $x$-axis is the solution set
				to the equation $y=mx+c$.

			\item There are $m_{1},m_{2},c\in \R$ so that the $x$-axis (in
				$\R^{3}$) is the solution set to the equation $z=m_{1}x+m_{2}y+c$.

			\item A system of exactly one equation can have an empty solution set.
		\end{enumerate}
		\begin{solution}
			\begin{enumerate}
				\item False. A counterexample is given by
					\[
						% We would like to use the following systeme command, but for some reason it is erroring
						% so instead we hardcode the result as an array.
						%\systeme{x_1+x_2+x_3+x_4=1, x_1+x_2+x_3+x_4=2, x_1+x_2+x_3+x_4=3}.
						\left\{\begin{array}{r@{\mkern5mu}c@{\mkern5mu}r@{\mkern5mu}c@{\mkern5mu}r@{\mkern5mu}c@{\mkern5mu}r@{\mkern5mu}l}x_{1}&+&x_{2}&+&x_{3}&+&x_{4}&=1\\x_{1}&+&x_{2}&+&x_{3}&+&x_{4}&=2\\x_{1}&+&x_{2}&+&x_{3}&+&x_{4}&=3\end{array}\right..
					\]

				\item False. A counterexample is given by
					\[
						% We would like to use the following systeme command, but for some reason it is erroring
						% so instead we hardcode the result as an array.
						%\systeme{0x_1+0x_2=0, 0x_1+0x_2=1}.
						\left\{\begin{array}{r@{\mkern5mu}c@{\mkern5mu}r@{\mkern5mu}l}0x_{1}&+&0x_{2}&=0\\0x_{1}&+&0x_{2}&=1\end{array}\right..
					\]

				\item False. Assume the $y$-axis can be represented as the
					complete solution to $y=mx+c$ for some $m,c$. Since $(x,y)=(0
					,0)$ and $(x,y)=(0,1)$ are both on the $y$ axis, we know
					$0=0m+c$ and $1=0m+c$. This gives $0=1$, which is false. Therefore,
					there's no $m,c\in \R$ so that the $y$-axis is the solution
					set to the equation $y = mx + c$.

				\item True. Take $m=0$, $c=0$. The equation then becomes $y=0$. A
					complete solution to this equation is given by $\mat{t \\ 0}$
					($t\in \R$), which is exactly the $x$-axis.

				\item False. The $x$-axis in $\R^{3}$ can be described as
					$\Set*{\mat{x \\ 0 \\ 0}\in\R^3: x\in\R}$. Assume the $x$-axis
					can be represented as the complete solution to $z=m_{1}x+m_{2}
					y+c$ for some $m_{1},m_{2},c$. Since $(x,y,z)=(0,0,0)$ is on
					the $x$ axis, we know $c=0$. Since $(x,y,z)=(1,0,0)$ is on
					the $x$ axis, we know that $m_{1}=0$. The equation then
					becomes $z=m_{2}y$. However, for each choice of $m_{2}$, $x=0
					, y=1, z=m_{2}$ is a solution to the system which does not lie
					in the $x$-axis. Therefore, there's no there's no $m_{1}, m_{2}
					,c\in \R$ so that the $x$-axis is the solution set to the equation
					$z = m_{1}x + m_{2}y+ c$.

				\item True. An example is given by
					\[
						\systeme{0x+0y=1}.
					\]
			\end{enumerate}
		\end{solution}
	\end{problist}
\end{exercises}
