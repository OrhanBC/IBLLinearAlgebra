\label{MODULEBASIS}
Lines or planes through the origin can be written as spans
of their direction vectors. However, a line or plane that doesn't
pass through the origin cannot be written as a span---it must
be expressed as a \emph{translated} span.

\begin{center}
	\begin{tikzpicture}
		\begin{axis}[
		    anchor=origin,
		    disabledatascaling,
		    xmin=-2,xmax=4,
		    ymin=-1,ymax=3,
			xtick={-2,...,4},
		    x=1cm,y=1cm,
		    grid=both,
		    grid style={line width=.1pt, draw=gray!10},
		    %major grid style={line width=.2pt,draw=gray!50},
		    axis lines=middle,
		    minor tick num=0,
		    enlargelimits={abs=0.5},
		    axis line style={latex-latex},
		    ticklabel style={font=\tiny,fill=white},
		    xlabel style={at={(ticklabel* cs:1)},anchor=north west},
		    ylabel style={at={(ticklabel* cs:1)},anchor=south west}
		]

			\draw[myorange, very thick] (-4,-2) -- (6,3) node[pos=.35, below right, xshift=-7pt] {The span of $\vec d$};
			\draw[black, thick, dashed, ->] (0,0) -- (2,1) node[midway, below right] {$\vec d$};
			\draw[mygreen, thick] (-3,0) -- (5,4) node[near end, above left, xshift=8pt, yshift=-2pt] {Not a span, but a translated span};
		\end{axis}
	\end{tikzpicture}
\end{center}


There's something special about sets that can be expressed as 
(untranslated) spans. In particular, since a linear combination
of linear combinations is still a linear combination, a span
is \emph{closed} with respect to linear combinations. That is, 
by taking linear combinations of vectors in a span, you cannot
escape the span. In general, sets that have this property are called
\emph{subspaces}.

\SavedDefinitionRender{Subspace}

In the definition of a subspace, property (i) is called
	being \emph{closed with respect to vector addition} and
	property (ii) is called being \emph{closed with respect to
	scalar multiplication}.

Subspaces generalize the idea of \emph{flat spaces through the origin}. They include 
lines, planes, volumes and more.

\begin{example}
	Let $\mathcal V\subseteq \R^2$ be the complete solution to 
	$x+2y=0$.  Show that $\mathcal V$ is a subspace.
	
	Let $\vec u = \mat{u_1\\u_2}$ and 
	$\vec v = \mat{v_1\\v_2}$ be in $\mathcal V$, 
	and let $k$ be a scalar.
	
	By definition, we have 
	\[
	\begin{aligned}
		u_1+2u_2&=0 \\
		v_1+2v_2&=0
	\end{aligned}.
	\]
	
	We will show that $\mathcal V$ is nonempty and that
		(i) $\vec u + \vec v \in \mathcal V$; and
		(ii) $k\vec u \in \mathcal V$.
	
	First we will show (i).
	Observe that
	\[
		\vec u + \vec v = \mat{u_1+v_1\\u_2+v_2}
	\]
	and the coordinates of $\vec u+\vec v$ satisfy
	\[
		(u_1+v_1)+2(u_2+v_2)=
		(u_1+2u_2)+(v_1+2v_2)=
		0+0=0.
	\]
	Since the coordinates of $\vec u + \vec v$ satisfy the equation $x+2y=0$, 
	we have shown that $\vec u + \vec v \in \mathcal V$.
	
	Next we will show (ii).
	Observe that
	\[
		k\vec u = \mat{ku_1\\ku_2}
	\]
	and the coordinates of $k\vec u$ satisfy
	\[
		(ku_1)+2(ku_2) = 
		k(u_1+2u_2)=
		k0=0.
	\]
	And so, we have shown that $k\vec u \in \mathcal V$.
	
	Finally, since $\vec 0 = \mat{0\\0}$ satisfies $x+2y=0$, 
	we conclude that $\vec 0\in\mathcal V$
	and so $\mathcal V$ is non-empty.
	
	Thus, by the definition, we have shown that $\mathcal V$ is a subspace.	
\end{example}

\begin{example}
	Let $\mathcal W\subseteq \R^2$ be the line expressed in vector form
	as 
	\[
		\vec x=t\mat{1\\2}+\mat{1\\1}.
	\]
	Determine whether $\mathcal W$ is a subspace.

	$\mathcal W$ is \emph{not} a subspace. To see this, notice that $\vec v=\mat{1\\1}\in \mathcal W$,
	but $0\vec v=\vec 0\notin\mathcal W$. Therefore, $\mathcal W$ is not closed under scalar multiplication
	and so cannot be a subspace.
\end{example}


As mentioned earlier, subspaces and spans are deeply connected.
This connection is given by the following theorem.

\begin{theorem}[Subspace-Span]
	Every subspace is a span and every span is a subspace.  More precisely,
	$\mathcal V\subseteq \R^n$ is a subspace if and only if $\mathcal V=
	\Span\mathcal X$ for some set $\mathcal X$.
\end{theorem}
\begin{proof}
	We will start by showing every span is a subspace.  Fix $\mathcal X\subseteq\R^n$
	and let
	$\mathcal V=\Span\mathcal X$. First note that if $\mathcal X\neq \Set{}$, then $\mathcal V$
	is non-empty because $\mathcal X\subseteq\mathcal V$ and if $\mathcal X=\Set{}$, then $\mathcal V=\Set{\vec 0}$,
	and so is still non-empty.
	
	Fix $\vec v,\vec u\in\mathcal V$. By definition there are $\vec x_1,\vec x_2,\ldots,\vec y_1,\vec y_2,\ldots\in\mathcal X$ 
	and scalars $\alpha_1,\alpha_2,\ldots,\beta_1,\beta_2,\ldots$ so that
	\[
		\vec v=\sum \alpha_i\vec x_i\qquad \vec u=\sum\beta_i\vec y_i.
	\]

	To verify property (i), observe that
	\[
		\vec u+\vec v=\sum\alpha_i\vec x_i + \sum\beta_i\vec y_i
	\]
	is also a linear combination of vectors in $\mathcal X$ (because all $\vec x_i$ and $\vec y_i$
	are in $\mathcal X$), and so $\vec u+\vec v\in	\Span\mathcal X=\mathcal V$.
	
	To verify property (ii), observe that for any scalar $\alpha$, 
	\[
		\alpha\vec v=\alpha\sum \alpha_i\vec x_i = \sum (\alpha\alpha_i)\vec x_i\in
		\Span\mathcal X=\mathcal V.
	\]
	Since $\mathcal V$ is non-empty and satisfies both properties (i) and (ii), it is a subspace.

	Now we will prove that every subspace is a span. Let $\mathcal V$ be a subspace
	and consider $\mathcal V'=\Span\mathcal V$.  Since taking a span may only enlarge a set, we know
	$\mathcal V\subseteq \mathcal V'$. If we establish that $\mathcal V'\subseteq\mathcal V$,
	then $\mathcal V=\mathcal V'=\Span\mathcal V$, which would complete the proof.

	Fix $\vec x\in\mathcal V'$. By definition, there are $\vec v_1,\vec v_2,\ldots,\vec v_n\in\mathcal V$ and scalars
	$\alpha_1,\alpha_2,\ldots,\alpha_n$ so that
	\[
		\vec x=\sum \alpha_i\vec v_i.
	\]
	Observe that 
	$\alpha_i\vec v_i\in\mathcal V$ for all $i$, since $\mathcal V$ is closed under scalar
	multiplication. It follows that $\alpha_1\vec v_1+\alpha_2\vec v_2\in\mathcal V$,
	because $\mathcal V$ is closed under sums. Continuing, 
	$(\alpha_1\vec v_1+\alpha_2\vec v_2)+\alpha_3\vec v_3\in\mathcal V$ because
	$\mathcal V$ is closed under sums. Applying the principle of finite induction, we see
	\[
		\vec x=\sum \alpha_i\vec v_i
		=\Big(\big((\alpha_1\vec v_1+\alpha_2\vec v_2)
		+\alpha_3\vec v_3\big)+\cdots+\alpha_{n-1}\vec v_{n-1} \Big)
		+\alpha_n\vec v_n\in\mathcal V.
	\]
	Thus $\mathcal V'\subseteq\mathcal V$, which completes the proof.
\end{proof}

The previous theorem is saying that spans and subspaces are two ways of talking about the same
thing. Spans provide a \emph{constructive} definition of lines/planes/volumes/etc. through the origin. That is,
when you describe a line/plane/etc. through the origin as a span, you're saying ``this is a line/plane/etc. through the origin
because every point in it is a linear combination of \emph{these specific vectors}''. In contrast, subspaces provide a \emph{categorical}
definition of lines/planes/etc. through the origin. 
When you describe a line/plane/etc. through the origin as a subspace, 
you're saying ``this is a line/plane/etc. through the origin because these \emph{properties} are satisfied''\footnote{
Categorical definitions are useful when working with objects where it's hard to pin down exactly what the elements
inside are.}.


\begin{emphbox}[Takeaway]
	Spans and subspaces are two different ways of talking about the same objects: points/lines/planes/etc. through the origin.
\end{emphbox}

\Heading{Special Subspaces}
When thinking about $\R^n$, there are two special subspaces that are always available. The first is $\R^n$ itself.
$\R^n$ is obviously non-empty, and linear combinations of vectors in $\R^n$ remain in $\R^n$. The second is
the \emph{trivial subspace}, $\Set{\vec 0}$.

\SavedDefinitionRender{TrivialSubspace}

\begin{theorem}
	The trivial subspace is a subspace.
\end{theorem}
\begin{proof}
	First note that $\Set{\vec 0}$ is non-empty since $\vec 0\in\Set{\vec 0}$. Now, since
	$\vec 0$ is the only vector in $\Set{\vec 0}$, properties (i) and (ii) follow quickly:
	\[
		\vec 0+\vec 0=\vec 0\in \Set{\vec 0}
	\]
	and
	\[
		\alpha\vec 0=\vec 0\in\Set{\vec 0}.
	\]
\end{proof}

\Heading{Bases}

Let $\vec d=\mat{2\\1}$ and consider $\ell=\Span\Set{\vec d}$.

\begin{center}
	\begin{tikzpicture}
		\begin{axis}[
		    anchor=origin,
		    disabledatascaling,
		    xmin=-1,xmax=4,
		    ymin=-1,ymax=2,
			xtick={-2,...,4},
		    x=1cm,y=1cm,
		    grid=both,
		    grid style={line width=.1pt, draw=gray!10},
		    %major grid style={line width=.2pt,draw=gray!50},
		    axis lines=middle,
		    minor tick num=0,
		    enlargelimits={abs=0.5},
		    axis line style={latex-latex},
		    ticklabel style={font=\tiny,fill=white},
		    xlabel style={at={(ticklabel* cs:1)},anchor=north west},
		    ylabel style={at={(ticklabel* cs:1)},anchor=south west}
		]

			\draw[myorange, very thick] (-2,-1) -- (6,3) node[pos=.65, above left] {$\ell=\Span\Set{\vec d}$};
			\draw[black, thick, dashed, ->] (0,0) -- (2,1) node[midway, below right, yshift=2pt] {$\vec d$};
		\end{axis}
	\end{tikzpicture}
\end{center}

We know that $\ell$ is a subspace, and we defined $\ell$ as the span of $\Set{\vec d}$,
but we didn't have to define $\ell$ that way. We could have, for instance, defined $\ell=\Span\Set{\vec d,-2\vec d,\tfrac{1}{2}\vec d}$. However,
$\Span\Set{\vec d}$ is a simpler way to describe $\ell$ than $\Span\Set{\vec d,-2\vec d,\tfrac{1}{2}\vec d}$.
This property is general: the simplest descriptions of a line involve the span of only one vector.


Analogously, let $\mathcal P=\Span\Set{\vec d_1,\vec d_2}$ be the plane
through the origin with direction vectors $\vec d_1$ and $\vec d_2$. There
are many ways to write $\mathcal P$ as a span, but the simplest ones
involve exactly two vectors. The idea of a \emph{basis} comes
from trying to find the simplest description of a subspace.

\SavedDefinitionRender{Basis}

In short, a basis for a subspace is a linearly independent set that spans that
subspace.

\begin{example}
	\label{EXLINEBASIS}
	Let $\ell=\Span\Set*{\mat{1\\2},\mat{-2\\-4}, \mat{1/2\\1}}$. Find
	two different bases for $\ell$.
	
	We are looking for a set of linearly independent vectors that spans $\ell$.
	Notice that $\mat{1\\2} = -\tfrac{1}{2}\mat{-2\\-4} = 2\matc{1/2\\1}$.
	Therefore,
    \[
		\Span\Set*{\mat{1\\2}} =
		\Span\Set*{\mat{-2\\-4}} =
		\Span\Set*{\mat{1/2\\1}} =
		\Span\Set*{\mat{1\\2},\mat{-2\\-4}, \mat{1/2\\1}} =
		\ell.
	\]
	Because $\Set*{\mat{1\\2}}$ is linearly independent and spans $\ell$,
	we have that $\Set*{\mat{1\\2}}$ is a basis for $\ell$.
	Similarly, $\Set*{\matc{1/2\\1}}$ is another basis for $\ell$.
\end{example}

Unpacking the definition of basis a bit more, we can see that 
a basis for a subspace
 is a set of vectors
that is \emph{just the right size} to describe everything in the subspace.
It's not too big---because it is linearly independent, there are no
redundancies. It's not too small---because we require it to span the subspace\footnote{ 
If you're into British fairy tales, you might call a basis a \emph{Goldilocks set}.
}.

There are several facts everyone should know about bases:
\begin{enumerate}
	\item Bases are not unique. Every subspace (except the trivial subspace)
		has multiple bases.
	\item Given a basis for a subspace, every vector in the subspace can be written
		as a \emph{unique} linear combination of vectors in that basis.
	\item Any two bases for the same subspace have the same number of elements.
\end{enumerate}

You can prove the first fact by observing that if $\mathcal B=\Set{\vec b_1,\vec b_2,\ldots}$ is a basis
with at least one element\footnote{ The empty set is a basis for the trivial subspace.}, then
$\Set{2\vec b_1,2\vec b_2,\ldots}$ is a different basis. The second fact is a consequence of all bases 
being linearly independent. The third fact is less obvious and takes some legwork to prove, so we will accept it
as is.

\Heading{Dimension}

Let $\mathcal V$ be a subspace. Though there are many bases for $\mathcal V$, they all
have the same number of vectors in them. And, this number says something fundamental about $\mathcal V$:
it tells us the maximum number of linearly independent vectors that can simultaneously exist in $\mathcal V$.
We call this number the \emph{dimension} of $\mathcal V$.

\SavedDefinitionRender{Dimension}

This definition agrees with our intuition about lines and planes: the dimension of a line through $\vec 0$ is $1$, and
the dimension of a plane through $\vec 0$ is $2$. It even tells us the dimension of the single point $\Set{\vec 0}$
is $0$\footnote{ The dimension of a line, plane, or point
not through the origin is defined to be the dimension of the subspace obtained
when it is translated to the origin.}.

\begin{example}
	Find the dimension of $\R^2$.

	Since $\Set{\xhat, \yhat}$ is a basis for $\R^2$, we know $\R^2$ is
	two dimensional.
\end{example}

\begin{example}
	Let $\ell=\Span\Set*{\mat{1\\2},\mat{-2\\-4}, \mat{1/2\\1}}$.
	Find the dimension of the subspace $\ell$.

	This is the same subspace from the earlier example where we found
	$\Set*{\mat{1\\2}}$ and $\Set*{\matc{1/2\\1}}$
	were bases for $\ell$. Both these bases contain one element, and so
	$\ell$ is a one dimensional subspace.
\end{example}

\begin{example}
	Let $A=\Set{(x_1,x_2,x_3,x_4)\given x_1+2x_2-x_3=0\text{ and }x_1+6x_4=0}$. Find a basis for and the dimension
	of $A$.

	$A$ is the complete solution to the system
	\[
		\systeme{x_1+2x_2-x_3=0,x_1+6x_4=0},
	\]
	which can be expressed in vector form as
	\[
		\mat{x_1\\x_2\\x_3\\x_4} = t\matc{0\\1/2\\1\\0}+s\mat{-6\\3\\0\\1}.
	\]
	Therefore $A=\Span\Set*{\matc{0\\1/2\\1\\0}, \mat{-6\\3\\0\\1}}$. Since $\Set*{\matc{0\\1/2\\1\\0}, \mat{-6\\3\\0\\1}}$
	is a linearly independent spanning set with two elements, $A$ is two dimensional.
\end{example}

Like $\R^2$ and $\R^3$, whenever we discuss $\R^n$, we always have a standard basis that comes
along for the ride.

\SavedDefinitionRender{StandardBasisforRn}

Note: the notation $\vec e_i$ is context specific. If we say $\vec e_i\in\R^2$,
then $\vec e_i$ must have exactly two coordinates. If we say $\vec e_i\in\R^{45}$,
then $\vec e_i$ must have $45$ coordinates. 
