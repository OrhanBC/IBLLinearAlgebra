\begin{exercises}
		% Topics:
		% Sets, set builder notation, set operations,
		% vectors \& scalars, vector notation, vectors \& points, vector arithmetic,
		% coordinates \& the standard basis, higher dimensions,
	\begin{problist}
		% Computation (4 questions)
		\prob
		\begin{enumerate}
			\item
			Write the following vectors as column vectors.
			\begin{enumerate}
				\item $4\xhat -3\zhat +2\yhat -2\xhat\in\R^3$.
				\item $\yhat +\xhat -5\yhat \in\R^2$.
			\end{enumerate}
			\item
			Write the following vectors as linear combinations of
			$\xhat$, $\yhat$, and $\zhat$.
			\begin{enumerate}
				\item $\mat{1\\-2\\3}$.
				\item $\mat{-2\\5\\4} + \mat{1\\-2\\ -5} + \mat{1\\0\\1}$.
			\end{enumerate}
		\end{enumerate}
		% Q1 Solution
		\begin{solution}
			\begin{enumerate}
				\item
				\begin{enumerate}
					\item $\mat{2\\2\\-3}$
					\item $\mat{1\\-4}$
				\end{enumerate}
				\item
				\begin{enumerate}
					\item $\xhat - 2\yhat + 3\zhat$
					\item $3\yhat$
				\end{enumerate}
			\end{enumerate}
		\end{solution}

		\prob
		Compute
		\[
			3\mat{2\\-1\\1\\1\\0}+
			(-2)\mat{1\\2\\-7\\3\\0}+
			\mat{-3\\3\\9\\2\\2}
		\]
		% Q2 Solutions
		\begin{solution}
		    $\mat{1\\-4\\26\\-1\\2}$
		\end{solution}

		\prob[\hefferon[2.21,2.22]]
		Decide if the vector is in the set. If it is, what value of the
		parameters produce that vector?
		\begin{enumerate}
			\item $\mat{5\\-5}$ and the set
			\[
				\Set*{\vec{v}\in\R^2 \given \vec{v}=k\mat{1\\-1} \text{ for some } k\in\R}
			\]
			\item $\mat{-1\\2\\1}$ and the set
			\[
				\Set*{\vec{v}\in\R^3 \given
				\vec{v}=i\mat{-2\\1\\0}+j\mat{3\\0\\1} \text{ for some } i,j\in\R}
			\]
			\item $\mat{3\\-1}$ and the set
			\[
				\Set*{\vec{v}\in\R^2 \given \vec{v}=k\mat{-6\\2} \text{ for some } k\in\R}
			\]
			\item $\mat{5\\4}$ and the set
			\[
				\Set*{\vec{v}\in\R^2 \given \vec{v}=j\mat{5\\-4} \text{ for some } j\in\R}
			\]
			\item $\mat{2\\1\\-1}$ and the set
			\[
				\Set*{\vec{v}\in\R^3 \given \vec{v}=r\mat{1\\-1\\3}+\mat{0\\3\\-7} \text{ for some } r\in\R}
			\]
			\item $\mat{1\\0\\1}$ and the set
			\[
				\Set*{\vec{v}\in\R^3 \given \vec{v}=j\mat{2\\0\\1}+k\mat{-3\\-1\\1} \text{ for some } j,k\in\R}
			\]
		\end{enumerate}
		% Q3 solution
		\begin{solution}
			\begin{enumerate}
		        \item Yes, take $k=5$.
		        \item Yes, take $i=2,j=1$.
		        \item Yes, take $k=-\frac{1}{2}$.
		        \item No.
		        \item Yes, take $r=2$.
		        \item No.
		    \end{enumerate}
		\end{solution}

		% Conceptual (3 questions)
		\prob
		% Purpose: get students to understand (the beginnings of) scale-invariance
		% of bases, as well as carefully reading mathematical expressions.
		Draw the following subsets of $\R^2$ and then determine which are equal or subsets of each other.
		\begin{enumerate}
			\item $A=\Set*{\vec v\in\R^2\given \vec v=n\mat{2\\1}\text{ for some integer }n\in\Z}$
			\item $B=\Set*{\vec v\in\R^2\given \vec v=t\mat{4\\2}\text{ for some }t\in\R}$
			\item $C=\Set*{\vec v\in\R^2\given \vec v=n\mat{4\\2}\text{ for some integer }n\in\Z}$
			\item $D=\Set*{\vec v\in\R^2\given \vec v=t\mat{2\\1}\text{ for some }t\in\R}$
		\end{enumerate}
		\begin{solution}
			\begin{enumerate}
				\item 
				\begin{tikzpicture}[baseline = (current bounding box.north)]
					\begin{axis}[
						anchor=origin,
						disabledatascaling,
						xmin=-4,xmax=4,
						ymin=-4,ymax=4,
						xtick={-4,-2,0,2,4},
						ytick={-4,-2,0,2,4},
						x=0.5cm,y=0.5cm,
						grid=both,
						grid style={line width=.1pt, draw=gray!10},
						axis lines=middle,
						minor tick num=0,
						enlargelimits={abs=1.0},
						axis line style={latex-latex},
						ticklabel style={font=\tiny,fill=white},
						xlabel style={at={(ticklabel* cs:1)},anchor=north west},
						ylabel style={at={(ticklabel* cs:1)},anchor=south west}
					]
					\end{axis}
					\foreach \n in {-2,...,2} {
						\fill [mypink] (\n,\n/2) circle (2pt);
					}
				\end{tikzpicture}
				\item 
				\begin{tikzpicture}[baseline = (current bounding box.north)]
					\begin{axis}[
						anchor=origin,
						disabledatascaling,
						xmin=-4,xmax=4,
						ymin=-4,ymax=4,
						xtick={-4,-2,0,2,4},
						ytick={-4,-2,0,2,4},
						x=0.5cm,y=0.5cm,
						grid=both,
						grid style={line width=.1pt, draw=gray!10},
						axis lines=middle,
						minor tick num=0,
						enlargelimits={abs=1.0},
						axis line style={latex-latex},
						ticklabel style={font=\tiny,fill=white},
						xlabel style={at={(ticklabel* cs:1)},anchor=north west},
						ylabel style={at={(ticklabel* cs:1)},anchor=south west}
					]
						\draw [mygreen, thick] (-5,-2.5) -- (5,2.5);
					\end{axis}
				\end{tikzpicture}
				\item 
				\begin{tikzpicture}[baseline = (current bounding box.north)]
					\begin{axis}[
						anchor=origin,
						disabledatascaling,
						xmin=-4,xmax=4,
						ymin=-4,ymax=4,
						xtick={-4,-2,0,2,4},
						ytick={-4,-2,0,2,4},
						x=0.5cm,y=0.5cm,
						grid=both,
						grid style={line width=.1pt, draw=gray!10},
						axis lines=middle,
						minor tick num=0,
						enlargelimits={abs=1.0},
						axis line style={latex-latex},
						ticklabel style={font=\tiny,fill=white},
						xlabel style={at={(ticklabel* cs:1)},anchor=north west},
						ylabel style={at={(ticklabel* cs:1)},anchor=south west}
					]
					\end{axis}
					\foreach \n in {-1,...,1} {
						\fill [mypink] (\n*2,\n) circle (2pt);
					}
				\end{tikzpicture}
				\item 
				The set $D$ and $B$ are equal.
			\end{enumerate}
			We have $C\subseteq A$, $C\subseteq B$, $C\subseteq D$,  $A\subseteq B$, $A\subseteq D$, and $B=D$.
		\end{solution}

		\prob
		Let $\vec a=\mat{1\\2}$, $\vec b=\mat{2\\4}$, $\vec c=\xhat+3\yhat$, and $\vec d=\vec a+\vec c$.
		\begin{enumerate}
			\item Is $\xhat$ a linear combination of $\vec a$ and $\vec b$?
			\item Is $\vec d$ a linear combination of $\vec a$ and $\vec b$?
			\item Is $\vec p=\mat{1\\1}$ a linear combination of $\vec a$ and $\vec c$?
			\item Is $\vec q=\mat{-3\\3}$ a linear combination of $\vec a$, $\vec b$, $\vec c$, and $\vec d$?
		\end{enumerate}
		% Q5 Solutions
		\begin{solution}
            \begin{enumerate}
    		    \item No.
    		    \item No.
    		    \item Yes.
    		    \item Yes.
		    \end{enumerate}
		\end{solution}
		
		\prob
		Use set-builder notation to describe the following sets.
		\begin{enumerate}
			\item The set of vectors in $\R^{2}$ whose coordinates are rational numbers.

			\item The set of vectors in $\R^{2}$ whose coordinates are irrational
				numbers.

			\item Let $P(\vec x) = \Set*{\vec x, -\vec x}$. The set $\Set*{P(\vec e_{1}), P(
        		\vec e_{2})}$.
		\end{enumerate}
		\begin{solution}
			\begin{enumerate}
				\item $\Set*{\vec{v}\in\R^2 \given \vec{v}=\mat{\alpha\\\beta} \text{ for some } \alpha,\beta\in\Q}$
				\item $\Set*{\vec{v}\in\R^2 \given \vec{v}=\mat{\alpha\\\beta} \text{ for some } \alpha,\beta\in\R\setminus\Q}$
				\item $\Set*{\vec{v}\in\R^2 \given \vec{v}=\pm\vec{e}_1 \text{ or } \vec{v}=\pm\vec{e}_2}$
			\end{enumerate}
		\end{solution}

		% Challenge (3 questions)
		\prob % Propose: get students to understand quantifiers and set operations
		% practice providing justification to their intuitive guesses
		Which of the following statements are true about the set listed below? Justify your
		answers.
		\begin{enumerate}
			\item $\mathcal{Y}$, the $y$-axis in $\R^{3}$.
				\begin{enumerate}
					\item $\mathcal{Y}$ is a finite set.

					\item Let
						$\mathcal{A} = \Set*
						{\vec a \in \R^3 \given \vec a = \beta \vec v \text{ for some } \vec v \in \mathcal{Y}, \beta \in \R}$,
						then $\mathcal{A} \subseteq \mathcal{Y}$.

					\item For all vectors $\vec v \in \mathcal{Y}$, we have $\vec v \neq
						\vec 0$.

					\item For some vectors $\vec v \in \mathcal{Y}$, we have $\vec v
						\neq \vec 0$.

					\item For all vectors $\vec v \in \mathcal{Y}$, there exists a vector
						$\vec x \in \mathcal{Y}$ such that $\vec x + \vec v = \vec e_{2}$.

					\item There exists a vector $\vec x \in \mathcal{Y}$ such that for
						all vectors $\vec v \in \mathcal{Y}$, we have $\vec x + \vec v =
						\vec e_{2}$.
				\end{enumerate}

			\item $\mathcal{S}$, the set of vectors in $\R^{3}$ whose coordinates are $\pm3$.
				\begin{enumerate}
					\item $\mathcal{S}$ is a finite set.

					\item Let
						$\mathcal{A} = \Set*
						{\vec a \in \R^3 \given \vec a = \beta \vec v \text{ for some } \vec v \in \mathcal{S}, \beta \in \R }$,
						then $\mathcal{A} \subseteq \mathcal{S}$.

					\item For all vectors $\vec v \in \mathcal{S}$, we have $\vec v \neq
						\vec 0$.

					\item For some vectors $\vec v \in \mathcal{S}$, we have $\vec v
						\neq \vec 0$.

					\item For all vectors $\vec v \in \mathcal{S}$, there exists a vector
						$\vec x \in \mathcal{S}$ such that $\vec x + \vec v = \vec 0$.

					\item There exists a vector $\vec x \in \mathcal{S}$ such that for
						all vectors $\vec v \in \mathcal{S}$, we have $\vec x + \vec v =
						\vec 0$.
				\end{enumerate}
		\end{enumerate}
		\begin{solution}
			\begin{enumerate}
				\item 
				\begin{enumerate}
					\item False.
					\item True.
					\item False.
					\item True.
					\item True.
					\item False.
				\end{enumerate}
				\item 
				\begin{enumerate}
					\item True.
					\item False.
					\item True.
					\item True.
					\item True.
					\item False.
				\end{enumerate}
			\end{enumerate}
		\end{solution}

		\prob % Propose: Give students some practices on proving general statements about sets
		For each of the following statements, determine whether it is correct or not. If
		it is, prove it. Otherwise, give a counterexample.
		\begin{enumerate}
			\item If $A \subseteq B$, then $A \cap B = A$.

			\item If $B \subseteq A$, then $A \cap B = A$.

			\item If $A \subseteq B$, then $A \cap B \neq B$.

			\item If $B \subseteq A$, then $A \cap B \neq B$.

			\item If $C \subseteq A \cap B$, then $C \subseteq A$.

			\item If $C \subseteq A \cup B$, then $C \subseteq A$.

			\item If $C \subseteq A \cup B$ and $C \subseteq B$, then $A \cap B
				\subseteq C$.
		\end{enumerate}
		\begin{solution}
			\begin{enumerate}
				\item Correct.
				\item Incorrect.
				\item Incorrect.
				\item Incorrect.
				\item Correct.
				\item Incorrect.
				\item Incorrect.
			\end{enumerate}
		\end{solution}
	\end{problist}
\end{exercises}
