\begin{exercises}
	\begin{problist}
		\prob Determine whether the following linear transformations are
		one-to-one, onto, or both. As well, determine whether or not they are
		invertible. Justify your answers.
		\begin{enumerate}
			\item $\mathcal S:\R^{2}\to\R^{2}$, where $\mathcal S$
				is the linear transformation that doubles every
				vector.

			\item $\mathcal R:\R^{2}\to\R^{2}$, where $\mathcal R$
				the linear transformation that rotates every
				vector clockwise by $72^{\circ}$.

			\item $\mathcal P:\R^{2}\to\R^{2}$, where $\mathcal P$
				the linear transformation that projects every
				vector onto the $y$-axis.

			\item $\mathcal F:\R^{2}\to\R^{2}$, where $\mathcal F$
				is the linear transformation that reflects every
				vector over the line $y=x$.

			\item $\mathcal T:\R^{3}\to\R^{3}$, where $\mathcal T$ is the linear
				transformation induced by the matrix
				$M_{\mathcal T} = \mat{1&2&3\\4&5&6\\7&8&9}$.

			\item $\mathcal U:\R^{3}\to\R^{2}$, where $\mathcal U$ is the linear
				transformation induced by the matrix
				$M_{\mathcal U} = \mat{1&2&3\\3&4&5}$.
		\end{enumerate}

		\begin{solution}
	    \begin{enumerate}
				\item
				$\mathcal S$ is both one-to-one and onto.
				No two distinct vectors can be doubled to become the same vector,
				thus the transformation is one-to-one.
				Every vector is double of the vector that is half of itself,
				thus the transformation is onto.
				$\mathcal S$ is invertible, its inverse is the transformation that
				halves every vector.

				\item
				$\mathcal R$ is both one-to-one and onto.
				No two distinct vectors can be rotated by $72^{\circ}$ clockwise to
				become the same vector,
				thus the transformation is one-to-one.
				Every vector is the $72^{\circ}$ clockwise rotation of the vector that
				is the $72^{\circ}$ counter-clockwise rotation of itself,
				thus the transformation is onto.
				$\mathcal R$ is invertible, its inverse is the transformation that
				rotates every vector counter-clockwise by $72^{\circ}$.

				\item
				$\mathcal P$ is neither one-to-one nor onto.
				Every vector on the x-axis is sent to the origin so there are distinct
				inputs that do not map to distinct outputs,
				thus the transformation is not one-to-one.
				Vectors that do not lie on the y-axis are not in the range of the
				transformation,
				thus the transformation is not onto.
				$\mathcal P$ is not invertible.

				\item
				$\mathcal F$ is both one-to-one and onto.
				No two distinct vectors can be reflected to become the same vector,
				thus the transformation is one-to-one.
				Every vector is the reflection of the vector that is the reflection of
				itself,
				thus the transformation is onto.
				$\mathcal F$ is invertible, its inverse is itself---the reflection about
				the line $y=x$.

				\item
				$\mathcal T$ is neither one-to-one nor onto.
				The RREF of $M_{\mathcal T}$ is $\mat{1&0&-1\\0&1&0\\0&0&0}$.
				$\Rank(\mathcal T)$ = $\Rank(M_{\mathcal T})$ = 2.
				by the rank-nullity theorem the nullity is 1.
				An entire line of vectors is getting mapped to
				$\vec{0}$ so there are distinct inputs that do not map to distinct
				outputs,
				thus the transformation is not one-to-one.
				The range of $\mathcal T$ = the column space of $M_{\mathcal T}$
				which is not all of $\R^{3}$,
				Thus the transformation is not onto.
				$\mathcal T$ is not invertible.

				\item
				$\mathcal U$ is onto but is not one-to-one.
				Because $\mathcal U$ maps from $\R^{3}$ to $\R^{2}$, the rank is at most
				2, so by the Rank-Nullity Theorem, the nullity must be at least 1. More
				than one vector is getting mapped to $\vec{0}$ so there are distinct
				inputs that do not map to distinct outputs,
				thus the transformation is not one-to-one.
				Since $\Rref(M_{\mathcal U})$ is $\mat{1&0&-1\\0&1&2}$.
				$\Rank(\mathcal U)=\Rank(M_{\mathcal U})=2$, which is
				the dimension of the codomain,
				thus the transformation is onto.
				$\mathcal U$ is not invertible.
			\end{enumerate}
		\end{solution}

		\prob Invert the following matrices or explain why they are not invertible.
		\begin{enumerate}
			\item $M_{1} = \mat{2&3\\1&-1}$

			\item $M_{2} = \mat{1&0\\1&0}$

			\item $M_{3} = \mat{0&2&1\\1&0&1\\-2&3&0}$

			\item $M_{4} = \mat{2&0&1&8\\1&-5&2&2\\3&-1&0&7}$

			\item $M_{5} = \mat{0&-3&1&2\\1&0&-1&1\\2&-1&0&0\\0&0&1&3}$
		\end{enumerate}

		\begin{solution}
	    \begin{enumerate}
				\item
				To find the inverse we row reduce $M_{1}$, applying each
				of the row operations to the identity matrix.

				$\left[\begin{array}{cc|cc}
							2&3&1&0\\
							1&-1&0&1\\
							\end{array}\right]$

				$\left[\begin{array}{cc|cc}
							1&\frac{3}{2}&\frac{1}{2}&0\\
							1&-1&0&1\\
							\end{array}\right]$

				$\left[\begin{array}{cc|cc}
							1&\frac{3}{2}&\frac{1}{2}&0\\
							0&-\frac{5}{2}&-\frac{1}{2}&1\\
							\end{array}\right]$

				$\left[\begin{array}{cc|cc}
							1&\frac{3}{2}&\frac{1}{2}&0\\
							0&1&\frac{1}{5}&-\frac{2}{5}\\
							\end{array}\right]$

				$\left[\begin{array}{cc|cc}
							1&0&\frac{1}{5}&\frac{3}{5}\\
							0&1&\frac{1}{5}&-\frac{2}{5}\\
							\end{array}\right]$

				So $M_{1}^{-1}=\mat{\frac{1}{5}&\frac{3}{5}\\
								 \frac{1}{5}&-\frac{2}{5}\\}$.
				\item $M_{2}$ does not row reduce to the identity matrix so it is not invertible.
				\item $M_{3}^{-1}=\mat{3&-3&-2\\2&-2&-1\\-3&4&2}$
				\item $M_{4}$ is not invertible because it is not a square matrix.
				\item $M_{5}^{-1}=\frac{1}{23}\mat{-4&-1&12&3\\-8&-2&1&6\\-3&-18&9&8\\1&6&-3&5}$
			\end{enumerate}
		\end{solution}

		\prob Solve the following systems in two ways: (i) by using row reduction,
		and (ii) by using inverse matrices.
		\begin{enumerate}
			\item $\systeme{2x+y=5,3x+7y=3}$

			\item $\systeme{2x+2y+3z=4,2x+2y+z=0, 4x+5y+6z=2}$
		\end{enumerate}

		\prob Let $A=\mat{1&3&5\\0&2&0\\-1&0&2}$.
		\begin{enumerate}
			\item Express $A^{-1}$ as the product of elementary matrices.

			\item Express $A$ as the product of elementary matrices.
		\end{enumerate}

		\prob For each statement below, determine whether it is true or false. Justify your answer.
		\begin{enumerate}
			\item For an arbitrary linear transformation $T:\R^{m}\to\R^{n}$,
				if $n \neq m$, then the linear transformation is
				not invertible.
			\item The matrix $M = \mat{1&0&0\\1&1&0}$ is an elementary
				matrix.
			\item Every elementary matrix is invertible.
			\item The product of elementary matrices is sometimes an elementary matrix.
			\item The product of elementary matrices is always an elementary matrix.
			\item A matrix that induces an invertible linear transformation is necessarily invertible.
			\item A transformation that is one-to-one and onto is always invertible.
			\item For two matricies $A$ and $B$, if $AB=I$, then $A$ and $B$ are invertible.
		\end{enumerate}

		\begin{solution}
			\begin{enumerate}
				\item True. If $m<n$ then $T$ cannot be onto. If $m>n$ then $T$ cannot
					be one-to-one. In both cases, $T$ cannot be invertible. We may conclude
					that the dimension of the domain and codomain must be equal in order
					for a transformation to be invertible.

				\item False. Elementary matricies are one row operation away from the identity
					matrix. The identity matrix of any size is always square. There is no
					identity matrix such that performing one row operation on it yields $M$.
					We may conclude that elementary matricies are always square.

				\item True. The inverse of an elementary matrix $E$ is another elementary
					matrix $E^{-1}$ which corresponds to the row operation that turns $E$
					into the identity matrix. $E^{-1}$ is the ``opposite" row operation.

				\item True. Let $E_{1}$ correspond to multiplying row 1 by 4. Let $E_{2}$
					correspond to multiplying row 1 by 2. The product $E_{2}E_{1}$ corresponds
					to multiplying row 1 by 8, which is also a single row operation and
					thus has a corresponding elementary matrix.

				\item False. Let $E_{1}$ correspond to multiplying row 1 by 4. Let $E
					_{2}$ correspond to multiplying row 2 by 2. The product $E_{2}E_{1}$
					corresponds to multiplying row 1 by 4 and row 2 by 2, which is not a
					single row operation and thus does not have a corresponding
					elementary matrix.

				\item True.

				\item True.

				\item False. Let $A = \mat{1&0&0\\0&1&0}$. Let $B = \mat{1&0\\0&1\\0&0}$.
					$AB = I$. $BA = \mat{1&0&0\\0&1&0\\0&0&0}\neq I$. $A$ and $B$ are not
					invertible. It is required that $AB = BA = I$ in order for matrices
					$A$ and $B$ to be invertible.
			\end{enumerate}
		\end{solution}
	\end{problist}
\end{exercises}
