
Associated with every linear transformation
are two specially named subspaces: the range and the null space.

\Heading{Range}
\SavedDefinitionRender{Range}

The range of a linear transformation has the exact same definition as the range of a function---it's the set of all outputs.
In other words, the range of a linear transformation is the \emph{image} of the entire domain with respect to that linear
transformation\footnote{ Some people say ``the \emph{image} of $T$'' as a short way of saying ``the image
of the entire domain of $T$ under $T$''. Used in this sense $\text{Image}(T)=\Range(T)$.}. However, unlike the range of an arbitrary function, the range of a linear transformation
is always a subspace.

\begin{theorem}
	Let $T:\R^n\to\R^m$ be a linear transformation. Then $\Range(T)\subseteq \R^m$ is a
	subspace\index{Subspace}.
\end{theorem}
\begin{proof}
	Since $\Range(T)=T(\R^n)$ and $\R^n$ is non-empty, we know that $\Range(T)$ is non-empty.
	Therefore, to show that $\Range(T)$ is a subspace, what remains to be shown is (i) that it's closed under vector addition,
	and (ii) that it is closed under scalar multiplication.
	\begin{enumerate}[label=(\roman*)]
		\item Let $\vec x,\vec y\in\Range(T)$.
			By definition, there exist $\vec u,\vec v\in\R^n$ such that $\vec x=T(\vec u)$
			and $\vec y=T(\vec v)$. Since $T$ is linear,
			\[
				\vec x+\vec y=T(\vec u)+T(\vec v)=T(\vec u+\vec v),
			\]
			and so $\vec x+\vec y\in\Range(T)$.
		\item Let $\vec x\in\Range(T)$ and let $\alpha$ be a scalar.
			By definition, there exists $\vec u\in\R^n$ such that $\vec x=T(\vec u)$,
			and so by the linearity of $T$,
			\[
			\alpha\vec x=\alpha T(\vec u)=T(\alpha\vec u).
			\]
			Therefore $\alpha\vec x\in\Range(T)$.
	\end{enumerate}
\end{proof}

When analyzing subspaces, we are often interested in how big they are. That information is captured by a number---the
\emph{dimension}\index{Dimension} of the subspace. For transformations, we also have a notion of how ``big'' they are, which is captured
in a number called the \emph{rank}.

\SavedDefinitionRender{RankofaLinearTransformation}

The rank of a linear transformation can be used to measure
its complexity or compressibility. A rank $0$ transformation must send all vectors to $\vec 0$. A rank
$1$ transformation must send all vectors to a line, etc.. So, by knowing just a single number---the rank---you
can judge how complicated the set of outputs of a linear transformation will be.

\begin{example}
	Let $\mathcal P$ be the plane given by $x+y+z=0$, and let $T:\R^3\to\R^3$ be projection onto $\mathcal P$.
	Find $\Range(T)$ and $\Rank(T)$.

	\medskip
	First we will find $\Range(T)$. Since $T$ is a projection onto $\mathcal P$, we know $\Range(T)\subseteq \mathcal P$.
	Because $T(\vec p)=\vec p$ for all $\vec p\in \mathcal P$, we know $\mathcal P\subseteq \Range(T)$, and so
	\[
		\Range(T)=\mathcal P.
	\]

	Since $\mathcal P$ is a plane, we know $\Dim(\mathcal P)=2=\Dim(\Range(T))=\Rank(T)$.
\end{example}

\Heading{Null Space}
The second special subspace is called the \emph{null space}.

\SavedDefinitionRender{NullSpace}

We've seen null spaces before. In the context of matrices when we asked questions like, ``Are these column vectors
linearly independent?'' Now that we understand linear transformation and subspaces,
we can consider this question anew.

Just like the range of a linear transformation, the null space of a linear transformation is always a subspace.

\begin{theorem}
	Let $T:\R^n\to\R^m$ be a linear transformation. Then $\Null(T)\subseteq \R^n$ is a subspace.
\end{theorem}
\begin{proof}
	Since $T$ is linear, $T(\vec 0)=\vec 0$ and so $\vec 0\in\Null(T)$ which shows that $\Null(T)$ is non-empty.
	Therefore, to show that $\Null(T)$ is a subspace, we only need to show (i) that it's closed under vector addition,
	and (ii) that it is closed under scalar multiplication.
	\begin{enumerate}[label=(\roman*)]
		\item Let $\vec x,\vec y\in\Null(T)$.
			By definition, $T(\vec x)=T(\vec y)=\vec 0$. By linearity we see
			\[
				T(\vec x+\vec y)=T(\vec x)+T(\vec y)=\vec 0+\vec 0=\vec 0,
			\]
			and so $\vec x+\vec y\in\Null(T)$.
		\item Let $\vec x\in\Null(T)$ and let $\alpha$ be a scalar.
			By definition, $T(\vec x)=\vec 0$,
			and so by the linearity of $T$,
			\[
			T(\alpha\vec x)=\alpha T(\vec x)=\alpha\vec 0=\vec 0.
			\]
			Therefore $\alpha\vec x\in\Null(T)$.
	\end{enumerate}
\end{proof}

Akin to the rank--range connection, there is a special number called the \emph{nullity} 
which specifies the dimension of the null space.

\SavedDefinitionRender{Nullity}

\begin{example}
	Let $\mathcal P$ be the plane given by $x+y+z=0$, and let $T:\R^3\to\R^3$ be projection onto $\mathcal P$.
	Find $\Null(T)$ and $\Nullity(T)$.

	\medskip
	First we will find $\Null(T)$. Since $T$ is a projection onto $\mathcal P$ (and because $\mathcal P$ passes
	through $\vec 0$), we know every normal vector for $\mathcal P$ will get sent to $\vec 0$ when $T$ is applied.
	And, besides $\vec 0$ itself, these are the only vectors that get sent to $\vec 0$. Therefore
	\[
		\Null(T)=\Set{\text{normal vectors}}\cup\Set{\vec 0}=\Span\Set*{\mat{1\\1\\1}}.
	\]
	Since $\Null(T)$ is a line, we know
	$
		\Nullity(T)=1.
	$
\end{example}

\Heading{Fundamental Subspaces of a Matrix}
Every linear transformation has a range and a null space. Analogously, every matrix is associated with
three fundamental subspaces.

\SavedDefinitionRender{FundamentalSubspaces}

Computationally, it's much easier to find the row space/column space/null space of a matrix than it is
to find the range/null space of a linear transformation because we can turn matrix questions into systems
of linear equations.

\begin{example}
	Find the null space of $M=\mat{1&2&5\\2&-2&-2}$.

	To find the null space of $M$, we need to solve the homogeneous matrix equation $M\vec x=\vec 0$.
	Row reducing, we see
	\[
		\Rref(M)=\mat{1&0&1\\0&1&2},
	\]
	and so the $z$ column is a free variable column. Therefore, the complete solution can be expressed in
	vector form as
	\[
		\mat{x\\y\\z}=t\mat{-1\\-2\\1},
	\]
	and so
	\[
		\Null(M)=\Span\Set*{\mat{-1\\-2\\1}}.
	\]
\end{example}

The column space and row space are just as easy to compute, since it just involves picking
a basis from the existing row or column vectors.

\begin{example}
	Let $M=\mat{1&2&5\\2&-2&-2}$. Find a basis for the row space and the column space of $M$.

	\medskip
	First the column space. We need to pick a basis for $\Span\Set*{\mat{1\\2},\mat{2\\-2},\mat{5\\-2}}$,
	which is the same thing as picking a maximal linearly independent subset of $\Set*{\mat{1\\2},\mat{2\\-2},\mat{5\\-2}}$.

	Putting these vectors as columns in a matrix and row reducing, we see
	\[
		\Rref\left(\mat{1&2&5\\2&-2&-2}\right) = \mat{1&0&1\\0&1&2}.
	\]
	The first and second columns are the only pivot columns and so the first and second \emph{original} vectors
	form a maximal linearly independent subset. Thus,
	\[
		\Col(M) = \Span\Set*{\mat{1\\2},\mat{2\\-2}} = \R^2\qquad\text{and a basis is}\qquad \Set*{\mat{1\\2},\mat{2\\-2}}.
	\]

	To find the row space, we need to pick a basis for $\Span\Set*{\mat{1\\2\\5},\mat{2\\-2\\-2}}$. Repeating
	a similar procedure, we see
	\[
		\Rref\left(\mat{1&2\\2&-2\\5&-2}\right) = \mat{1&0\\0&1\\0&0},
	\]
	and so $\Set*{\mat{1\\2\\5},\mat{2\\-2\\-2}}$ is linearly independent. Therefore
	\[
		\Row(M)=\Span\Set*{\mat{1\\2\\5},\mat{2\\-2\\-2}}\qquad\text{and a basis is}\qquad \Set*{\mat{1\\2\\5},\mat{2\\-2\\-2}}.
	\]
\end{example}

When talking about fundamental subspaces, we often switch between talking about column vectors and
row vectors belonging to a matrix. The operation of swapping rows for columns is
called the \emph{transpose}.

\SavedDefinitionRender{Transpose}

Using the transpose, we can make statements like
\[
	\Col(M)=\Row(M^T)\qquad\text{and}\qquad \Row(M)=\Col(M^T).
\]

In addition, it helps us state the following theorem.
\begin{theorem}[Row-Col Dimension]
	For a matrix $A$, the dimension of the row space equals the dimension of the column space. 
	%That is,
	%$\Rank(A)=\Rank(A^T)$.
\end{theorem}
\begin{proof}
	For this proof, we will rely on what we know about the row reduction algorithm and what
	the reduced row echelon form\index{Matrix!reduced row echelon form} of a matrix tells us.

	\emph{Claim 1}: $\Row(\Rref(A))\subseteq \Row(A)$. To see this, observe that to get $\Rref(A)$, we take
	linear combinations of the rows of $A$. Therefore, it must be that the span of the rows of $\Rref(A)$
	is contained in the span of the rows of $A$.

	\emph{Claim 2}: $\Row(\Rref(A))=\Row(A)$. To see this, observe that every elementary row operation is
	reversible. Therefore every row in $A$ can be obtained as a linear combination of rows in $\Rref(A)$
	(by just reversing the steps). Thus the row vectors of $\Rref(A)$ and the row vectors of $A$ must
	have the same span.

	\emph{Claim 3}: The non-zero rows of $\Rref(A)$ form a basis for $\Row(A)$. We already know that
	the non-zero rows of $\Rref(A)$ span $\Row(A)$, so we only need to argue that they are linearly
	independent. However, this follows immediately from the fact that $\Rref(A)$ is in reduced row
	echelon form. Above and below every pivot in $\Rref(A)$ are zeros. Therefore, a row in $\Rref(A)$
	with a pivot cannot be written as a linear combination of any other row. Since every non-zero
	row has a pivot, this proves the claim.

	Now, note the following two facts.
	\begin{enumerate}
		\item The columns of $A$ corresponding to pivot columns of $\Rref(A)$ form a basis for $\Col(A)$.
		\item The non-zero rows of $\Rref(A)$ form a basis for $\Row(A)$.
	\end{enumerate}

	To complete the proof, note that every pivot of $\Rref(A)$ lies in exactly one row and one column. Therefore,
	the number of basis vectors in $\Row(A)$ is the same as the number of basis vectors in $\Col(A)$. 
	%Thus
	%$\Rank(A)=\Rank(A^T)$.
\end{proof}

\Heading{Equations, Null Spaces, and Geometry}

Let $M=\mat{1&2&5\\2&-2&-2}$. Using the typical row-reduction steps, we know that the complete solution to $M\vec x=\vec 0$
(i.e., the null space of $M$) can be expressed in vector form as
\[
	\vec x=t\mat{-1\\-2\\1}.
\]
Similarly, the complete solution to $M\vec x=\vec b$ where $\vec b=\mat{1\\2}$ can be expressed in vector form as
\[
	\vec x=t\mat{-1\\-2\\1}+\mat{1\\0\\0}.
\]

The set of solutions to $M\vec x=\vec 0$ and $M\vec x=\vec b$ look very similar. In fact,
\[
	\Set{\text{solutions to $M\vec x=\vec b$}} = \Set{\text{solutions to $M\vec x=\vec 0$}}+\Set{\vec p}\qquad\text{where}
	\qquad \vec p=\mat{1\\0\\0}.
\]
Or, phrased another way, the solution set to $M\vec x=\vec b$ is
\[
	\Null(M)+\Set{\vec p}.
\]

In the context of what we already know about lines and translated spans, this makes perfect sense. We know
that the solution set to $M\vec x=\vec b$ is a line (which doesn't pass through the origin) and may therefore
be written as a translated span $\Span\Set{\vec d}+\Set{\vec p}$. Here $\vec d$ is a direction vector for the line
and $\vec p$ is a point on the line.

Because $\vec p\in\Span\Set{\vec d}+\Set{\vec p}$, we call $\vec p$ a \emph{particular solution} to $M\vec x=\vec b$.
 Using a similar argument, we can show that for any matrix $A$, and any vector $\vec b$, the set of all solutions
to $A\vec x=\vec b$ (provided there are any) can be expressed as
\[
	V+\Set{\vec p}
\]
where $V$ is a subspace and $\vec p$ is a particular solution. In fact, we can do better. We can say $V=\Null(A)$.

\begin{theorem}
	Let $A$ be a matrix, $\vec b$ be a vector, and let $\vec p$ be a particular solution
	to $A\vec x=\vec b$. Then, the set of all solutions to $A\vec x=\vec b$ is
	\[
		\Null(A)+\Set{\vec p}.
	\]
\end{theorem}
\begin{proof}
	Let $S=\Set{\text{all solutions to $A\vec x=\vec b$}}$ and assume $\vec p\in S$. We will show $S=\Null(A)+\Set{\vec p}$.

	First we will show $\Null(A)+\Set{\vec p}\subseteq S$. Let $\vec v\in \Null(A)+\Set{\vec p}$. By definition,
	$\vec v=\vec n+\vec p$ for some $\vec n\in \Null(A)$. Now, by linearity of matrix multiplication
	and the definition of the null space,
	\[
		A\vec v=A(\vec n+\vec p)=A\vec n+A\vec p=\vec 0+\vec b=\vec b,
	\]
	and so $\vec v\in S$.

	Next we will show $S\subseteq \Null(A)+\Set{\vec p}$. First observe that for any
	$\vec u,\vec v\in S$ we have
	\[
		A(\vec u-\vec v)=A\vec u-A\vec v=\vec b-\vec b=\vec 0,
	\]
	and so $\vec u-\vec v\in \Null(A)$.

	Fix $\vec w\in S$. By our previous observation, $\vec w-\vec p\in \Null(A)$. Therefore
	\[
		\vec w=(\vec w-\vec p)+\vec p\in \Null(A)+\Set{\vec p},
	\]
	which completes the proof.
\end{proof}


\begin{emphbox}[Takeaway]
	To write the complete solution to $A\vec x=\vec b$, all you need is the
	null space of $A$ and a particular solution to $A\vec x=\vec b$.
\end{emphbox}


\bigskip
Null spaces are also closely connected with row spaces. Let $\mathcal P\subseteq \R^3$ be the plane
with equation $x+2y+2z=0$. We can rewrite this equation as a matrix equation and as the equation
of a plane in normal form.
\[
	\underbrace{\mat{1&2&2}\mat{x\\y\\z}=\vec 0}_{\text{a matrix equation}}\qquad\qquad
	\underbrace{\mat{1\\2\\2}\cdot \mat{x\\y\\z}=0}_{\text{normal form}}
\]
Now we see that $\mathcal P=\Null\left(\mat{1&2&2}\right)$ and that every non-zero vector in $\Row\left(\mat{1&2&2}\right)$
is a normal vector for $\mathcal P$. In other words, $\Null\left(\mat{1&2&2}\right)$ is orthogonal to $\Row\left(\mat{1&2&2}\right)$.

This is no coincidence. Let $M$ be a matrix and let $\vec r_1,\ldots,\vec r_n$ be the rows of $M$. By definition,
\[
	M\vec x=\matc{\vec r_1\cdot \vec x\\\vdots\\\vec r_n\cdot \vec x},
\]
and so solutions to $M\vec x=\vec 0$ are precisely the vectors which are orthogonal to every row of $M$. In other words,
$\Null(M)$ consists of all vectors orthogonal to the rows of $M$.
Conversely, $\Row(M)$ consists of all vectors orthogonal to everything in $\Null(M)$. We can use this
fact to approach questions in a new way.

\begin{example}
	Let $\vec a=\mat{1\\2\\5}$ and $\vec b=\mat{2\\-2\\-2}$. Find the set of all vectors orthogonal
	to both $\vec a$ and $\vec b$.

	Let $M=\mat{1&2&5\\2&-2&-2}$ be the matrix whose rows are $\vec a$ and $\vec b$. Since
	$\Null(M)$ consists of all vectors orthogonal to $\Row(M)$, the set we are looking for is
	$\Null(M)$. Computing via row reduction, we find
	\[
		\Null(M)=\Span\Set*{\mat{-1\\-2\\1}}.
	\]
\end{example}



\Heading{Transformations and Matrices}

Matrices are connected to systems of linear equations via matrix equations (like $A\vec x=\vec b$)
and to linear transformations through matrix transformations (like $\mathcal T(\vec x)=M\vec x$).
This means that we can think about systems of equations in terms of linear transformations and
we can gain insight about linear transformations by looking at systems of equations!

In preparation for this, let's reconsider matrix transformations and be pedantic about our notation.

Let $\mathcal T:\R^n\to\R^n$ be a linear transformation and let $M$ be its corresponding matrix.
$\mathcal T$ is a function that inputs and outputs \emph{vectors}. $M$ is a box of numbers,
which has no meaning by itself, but we know how to multiply $M$ by lists of numbers (or other boxes of numbers).
Therefore, strictly speaking, the expression ``$M\vec x$'' doesn't make sense. The quantity ``$\vec x$'' is a
vector, but we only know how to multiply $M$ by lists of numbers.

Ah! But we know how to turn $\vec x$ into a list of numbers. Just pick a basis! The expression
\[
	M[\vec x]_{\mathcal E}
\]
makes perfect sense since $[\vec x]_{\mathcal E}$ is a list of numbers. Continuing to be pedantic, we know
$\mathcal T(\vec x)\neq M[\vec x]_{\mathcal E}$ since the left side is a vector and the right side is a list of numbers.
We can fix this by either turning the right side into a vector or the left side into a list of numbers.
Doing this, we see the precise relationship between a linear transformation $\mathcal T:\R^n\to\R^n$ and its matrix $M$ is
\[
	[\mathcal T(\vec x)]_{\mathcal E}=M[\vec x]_{\mathcal E}.
\]

If we have a matrix $M$, by picking a basis (usually the standard basis),
we can define a linear transformation by first taking the input vector and rewriting it in the basis,
next multiplying by the matrix, and finally taking the list of numbers and using them as coefficients
for a linear combination involving the basis vectors. This is what we actually mean
when we say that a matrix \emph{induces} a linear transformation.

\SavedDefinitionRender{InducedTransformation}

Previously, we would write ``$\mathcal T(\vec x)=M\vec x$'' which hides the fact that when we relate a matrix and a linear
transformation, there is a basis hidden in the background. And, like before, when we're only considering a single basis,
we can be sloppy with our notation and write things like ``$M\vec x$'', but when there are multiple bases or when we're
trying to be extra precise, we must make sure our boxes/lists of numbers and our transformations/vectors stay separate.

\begin{example}
	Let $\mathcal T$ be the transformation induced by the matrix $M=\mat{1&2&5\\2&-2&-2}$,
	and let $\vec v=3\xhat-3\zhat$. Compute $\mathcal T(\vec v)$.

	Since $\mathcal T$ is induced by $M=\mat{1&2&5\\2&-2&-2}$, by definition,
	\[
	    [\mathcal T_{M}\vec v]_{\mathcal E'}= M[\vec v]_{\mathcal E}=\mat{1&2&5\\2&-2&-2}[\vec v]_{\mathcal E}.
	\]
	Further, since $\vec v=3\xhat-3\zhat$, by definition we have $[\vec v]_{\mathcal E}=\mat{3\\0\\-3}$. Therefore,
	\[
	    [\mathcal T_{M}\vec v]_{\mathcal E'}=\mat{1&2&5\\2&-2&-2}\mat{3\\0\\-3}=\mat{-12\\12}.
	\]
	In other words, $\mathcal T(\vec v) = \mat{-12\\12}_{\mathcal E'}=-12\xhat+12\yhat$.
\end{example}

Using induced transformations, we can extend linear-transformation definitions to matrix definitions.
In particular, we can define the rank and nullity of a matrix.

\SavedDefinitionRender{RankofaMatrix}
\SavedDefinitionRender{NullityofaMatrix}

\Heading{Range vs\mbox{.} Column Space \& Null Space vs\mbox{.} Null Space}
Let $M=\mat{C_1&C_2&\cdots &C_m}$ be an $m\times m$ matrix with columns $C_1$, \ldots, $C_m$, and let $\mathcal T$ be
the transformation induced by $M$. The column space of $M$ is the set of all linear combinations of the columns of
$M$. But, let's be precise. The columns of $M$ are lists of numbers, so to talk about the column space of $M$,
we need to turn them into vectors. Fortunately, we have a nice notation for that. Since $C_i$ is a list of numbers,
$[C_i]_{\mathcal E}$ is a (true) vector, and
\[
	\Col(M)=\Span\Set*{\hspace{1pt}[C_1]_{\mathcal E},[C_2]_{\mathcal E},\ldots,[C_m]_{\mathcal E}}.
\]

Can we connect this to the range of $\mathcal T$? Well, by the definition of matrix multiplication, we know that
\[
	M\matc{1\\0\\\vdots\\0}=M[\vec e_1]_{\mathcal E} = C_1
\]
and in general $M[\vec e_i]_{\mathcal E}=C_i$. By the definition of induced transformation, we know
\[
	[\mathcal T(\vec e_i)]_{\mathcal E} = M[\vec e_i]_{\mathcal E} = C_i,
\]
and so
\[
	\mathcal T(\vec e_i)= [C_i]_{\mathcal E}.
\]
Every input to $\mathcal T$ can be written as a linear combination of $\vec e_i$'s (because $\mathcal E$ is a basis)
and so, because $\mathcal T$ is linear, every output of $\mathcal T$ can be written as a linear combination of $[C_i]_{\mathcal E}$'s.
In other words,
\[
	\Range(\mathcal T)=\Col(M).
\]
This means that when trying to answer a question about the range of a linear transformation, we could think about
the column space of its matrix instead (or vice versa).

\begin{example}
	Let $\mathcal T:\R^3\to\R^2$ be defined by
	\[
		\mathcal T\mat{x\\y\\z} = \matc{2x-z\\4x-2z}.
	\]
	Find $\Range(\mathcal T)$ and $\Rank(\mathcal T)$.

	\medskip
	Let $M$ be a matrix for $\mathcal T$. We know $\Range(\mathcal T)=\Col(M)$ and $\Rank(\mathcal T)=\Dim(\Range(\mathcal T))
	=\Dim(\Col(M))$. By inspection, we see that
	\[
		M=\mat{2&0&-1\\4&0&-2}.
	\]
	Again, by inspection, we see that $\Set*{\mat{2\\4}}$ is a basis for $\Col(M)$ and $\Col(M)$ is one dimensional.
	Therefore,
	\[
		\Range(\mathcal T) = \Span\Set*{\mat{2\\4}}\qquad \text{and}\qquad \Rank(\mathcal T)=1.
	\]
\end{example}

There is an alternative definition of the rank of a matrix which commonly appears.
We'll state it as a theorem.

\begin{theorem}
	Let $M$ be a matrix. The rank of $M$ is equal to the number
	of pivots in $\Rref(M)$.
\end{theorem}
\begin{proof}
	We know that $\Rank(M)=\Dim(\Range(\mathcal T_M))=\Dim(\Col(M))$ where $\mathcal T_M$ is the transformation
	induced by $M$. Further, a basis for $\Col(M)$ consists of a maximal linearly independent subset
	of the columns of $M$. To find such a subset, we row reduce $M$ and look at the columns of $M$ that correspond
	to pivot columns of $\Rref(M)$.

	When all is said and done, the number of elements in a basis for $\Col(M)$ will be the number of pivots in $\Rref(M)$,
	which is the same as $\Rank(M)$.
\end{proof}

\begin{emphbox}[Takeaway]
	If $\mathcal T$ is a linear transformation and $M$ is a corresponding
	matrix, $\Range(\mathcal T)=\Col(M)$, and answering questions about $M$ answers
	questions about $\mathcal T$.
\end{emphbox}

Just like the range--column-space relationship, we also have a null-space--null-space relationship.
More specifically, if $\mathcal T$ is a linear transformation with matrix $M$, then
$\Null(\mathcal T)=\Null(M)$. From this fact, we deduce the following theorem.

\begin{theorem}
	Let $\mathcal T$ be a linear transformation and let $M$ be a matrix for $\mathcal T$. Then
	$\Nullity(\mathcal T)$ is equal to the number of free variable columns in $\Rref(M)$.
\end{theorem}
\begin{proof}
	We know $\Nullity(\mathcal T)=\Dim(\Null(\mathcal T))=\Dim(\Null(M))$. Further, we know
	that the complete solution to $M\vec x=\vec 0$ will take the form
	\[
		\vec x=t_1\vec d_1+\cdots +t_k\vec d_k
	\]
	where $k$ is the number of free variable columns in $\Rref(M)$. The algorithm for
	writing the complete solution to $M\vec x=\vec 0$ ensures that $\Set{\vec d_1,\ldots,\vec d_k}$
	is a basis for $\Null(M)$, and so $\Nullity(\mathcal T)=k$, which completes the proof.
\end{proof}

Combining these facts, we can reformulate the Row-Col Dimension theorem as a theorem
about ranks.
\begin{theorem}
	For a matrix $A$, we have $\Rank(A)=\Rank(A^T)$.
\end{theorem}
\begin{proof}
	By the Row-Col Dimension theorem, we know $\dim(\Col(A))=\dim(\Row(A))$. By the definition
	of the transpose, we know $\Row(A)=\Col(A^T)$. Therefore,
	\[
		\dim(\Col(A))=\dim(\Col(A^T)),
	\]
	which is another way of saying $\Rank(A)=\Rank(A^T)$.
\end{proof}


\Heading{The Rank-Nullity Theorem}

The rank and the nullity of a linear transformation/matrix are connected by a powerful theorem.

\begin{theorem}[Rank-nullity Theorem for Matrices]\index{Rank-nullity theorem!for matrices} For a matrix $A$,
\[
	\Rank(A)+\Nullity(A)=\ \#\text{ of columns in $A$}.
\]
\end{theorem}

The rank-nullity theorem's statement is simple, but it is surprisingly useful. Consider
the matrix \[M=\mat{1&2&2}.\] We already know that $\Null(M)$ is the plane with equation
$x+2y+2z=0$ and therefore is two dimensional. Since $M$ has three columns, $\Rank(M)=1$, and
so $\Dim(\Col(M))=\Dim(\Row(M))=1$, which means that $M$ has a one-dimensional set of normal vectors
(if we include $\vec 0$).

By contrast, let $\mathcal P\subseteq \R^4$ be the plane in $\R^4$ given in vector form by
\[
	\vec x=t\mat{1\\2\\2\\2}+s\mat{-1\\1\\-1\\1}.
\]
How many normal vectors does $\mathcal P$ have? Well, the matrix $A=\mat{1&2&2&2\\-1&1&-1&1}$ is rank $2$,
and so has nullity $2$. Therefore, there exist two linearly independent normal directions for $\mathcal P$.

\medskip
There is an equivalent rank-nullity theorem for linear transformations.
\begin{theorem}[Rank-nullity Theorem for Linear Transformations]\index{Rank-nullity theorem!for linear transformations} Let $\mathcal T$ be a linear transformation.
Then
\[
	\Rank(\mathcal T)+\Nullity(\mathcal T)=\Dim(\text{domain of $\mathcal T$}).
\]
\end{theorem}

Just like the rank-nullity theorem for matrices, the rank-nullity theorem for linear transformations
can give insights about linear transformations that would be otherwise hard to see.
