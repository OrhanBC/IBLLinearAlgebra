	An \emph{equation} encodes a relationship between quantities. For
	example, writing
	\[
		\underbrace{\text{Slices of cake}}_C = \underbrace{\text{Slices you ate}}_M + \underbrace{\text{Slices your brother ate}}_B
	\]
	specifies a precise relationship between the quantities $C$, $M$, and $B$. 
	Without more information, $C$, $M$, and $B$ could be almost anything. As such, we call 
	$C$, $M$, and $B$
	\emph{variables} or \emph{unknowns}. 
	However, the
	relationship \emph{between} them is precisely defined. 

	Additional relationships give rise to additional equations, which we express concisely as a
	\emph{system of equations}, that is, a list of equations. For example, suppose you know the cake
	had six pieces and your brother ate twice as many pieces as you. We might now write the system
	\[
		\sysdelim..
		\systeme{C=M+B,B=2M,C=6}
	\]
	which should be interpreted as: ``the relationship $C=M+B$ holds \emph{and} the relationship $B=2M$ holds \emph{and}
	the relationship $C=6$ holds.''
	All this information, taken together, is enough to deduce the unknowns: $M=2$, $B=4$, and $C=6$.

	Systems of equations naturally appear in linear algebra through vector equations. Suppose $\vec u=\mat{1\\2}$, $\vec v=\mat{2\\3}$,
	and $\vec w=\mat{1\\1}$. You might wonder if $\vec w$ was a linear combination of $\vec u$ and $\vec v$. The answer is \emph{yes} if and
	only if the vector equation
	\[
		\vec w=a\vec u+b\vec v
	\]
	has a solution for some $a$ and $b$. Written in coordinates, this equation is equivalent to
	\[
		\mat{1\\1}=a\mat{1\\2}+b\mat{2\\3} = \mat{a+2b\\2a+3b}.
	\]
	Equating coordinates, a system of equations appears:
	\[
		\systeme{a+2b=1,2a+3b=1}
	\]

	Every vector equation, by way of coordinates, corresponds to a system of equations. And, fortunately for us, there is
	an \emph{algorithm} to find all solutions to these systems\footnote{ Saying there is an \emph{algorithm} for ``$X$''
	means that there is a specific set of (non-random) rules that \emph{always} accomplishes ``$X$''. As a consequence, doing
	``$X$'' never requires special insight. For example, there \emph{is} an algorithm for multiplying numbers, but there \emph{is not}
	an algorithm for factoring polynomials of degree 5 or greater.}.

	\Heading{Systems of Linear Equations}

	There's no guarantee that a general equation, like $x^4-e^x+7=0$, has a solution, and it might
	be impossible to decide if an arbitrary equation has a solution, let alone what the solutions 
	are!\footnote{ Fermat's Last Theorem famously claimed that $a^n+b^n=c^n$
	has no positive integer solutions for $n\geq 3$. However, it took 350 years of human ingenuity before anyone was able rigorously
	back up the claim.} However, for \emph{linear} equations and systems of linear equations we can \emph{always} tell
	whether there is a solution and what the solution(s) are.
	
	\begin{definition}[Linear Equation]
		A \emph{linear equation} in the variables $x_1,\ldots,x_n$ is one that can be expressed
		as
		\[
			a_1x_1+a_2x_2+\cdots+a_nx_n=c
		\]
		for constants $a_1,\ldots, a_n$ and $c$. A \emph{system of linear equations} is a system of equations 
		consisting of one or more linear equations.
	\end{definition}

	Every vector equation corresponds to an \emph{equivalent} system of linear equations and vice versa, where equivalent
	means ``expresses the same relationships between variables''.

	\begin{example}
		Write down the vector equation corresponding to the system of linear equations $\systeme{2x+3y+z=2,y-z=-1}$
		and the system of linear equations corresponding to the vector equation $t\vec w+\vec u=r\vec v$ where $\vec w=\mat{1\\-1}$, $\vec u=\mat{2\\3}$,
		and $\vec v=\mat{4\\4}$.

		The system $\systeme{2x+3y+z=2,0x+y-z=-1}$ corresponds to the vector equation
		\[
			x\mat{2\\0} + y\mat{3\\1} + z\mat{1\\-1} = \mat{2\\-1}.
		\]
		
		As for the vector equation $t\vec w+\vec u=r\vec v$, rewriting each vector in coordinates gives us a corresponding system of linear equations:
		\begin{align*}
			t\vec w + \vec u = r\vec v
			\rightarrow
			t\mat{1\\-1} + \mat{2\\3} &= r\mat{4\\4}\\
			\mat{t+2\\-t+3} &= \mat{4r\\4r}
			\rightarrow
			\systeme{t-4r=-2, -t-4r=-3}.
		\end{align*}
	\end{example}

	\begin{emphbox}[Takeaway]
		Every vector equation corresponds to a system of linear equations and every system of linear equations
		corresponds to a vector equation.
	\end{emphbox}

	\Heading{Solution Sets}

	Before looking at how to solve systems of linear equations, let's agree on some terminology.

	A \emph{solution} to an equation is a particular choice of values for the variables that satisfy (i.e. make true)
	the equation. For example
	\begin{equation}
		\label{SOLNSETEXAMPLE}
		x+y=4
	\end{equation}
	has a solution $x=y=2$. However, $x=y=2$ is just one of \emph{many} possible solutions; we also have $x=4$ and $y=0$ or $x=-2$ and
	$y=6$. The \emph{solution set},
	also called the \emph{complete solution}, to an equation (or system of equations) is the set of all possible solutions.
	For example, the solution set to Equation \eqref{SOLNSETEXAMPLE} is $S=\Set{(x,y)\given y=4-x}$. %=\Set{(t,4-t)\given a\in \R}$.
	In this case, $S$ contains infinitely many solutions, including $(x,y)=(2,2)$, the first solution we found.

	Solution sets look a lot like sets of vectors: the set $S=\Set{(x,y)\given y=4-x}$ could be thought of as a subset of $\R^2$
	where we identify a solution $x=a$ and $y=b$ with the column vector $\mat{a\\b}$. Drawing $S$ as a subset of $\R^2$, we see a familiar
	picture.

	\begin{center}
		\begin{tikzpicture}
			\begin{axis}[
				anchor=origin,
				disabledatascaling,
				xmin=0,xmax=5,
				ymin=0,ymax=5,
				xtick={-1,...,5},
				ytick={-1,...,5},
				x=1cm,y=1cm,
				grid=both,
				grid style={line width=.1pt, draw=gray!10},
				axis lines=middle,
				minor tick num=0,
				enlargelimits={abs=0.5},
				axis line style={latex-latex},
				ticklabel style={font=\tiny,fill=white},
				xlabel style={at={(ticklabel* cs:1)},anchor=north west},
				ylabel style={at={(ticklabel* cs:1)},anchor=south west}
			]
				\draw [mygreen, thick] (-2,6) -- (6,-2);
			\end{axis}
		\end{tikzpicture}
	\end{center}

	It's the graph of the line given in $y=mx+b$ form by $y=-x+4$. In other words, \emph{via solution sets, equations and systems
	of equations can represent geometric objects.}

%	\medskip
%	If this all seems straightforward, great! Since grade school you've been training to apply algebra to geometry problems
%	and vice versa\footnote{ It was a big deal when equations saw their first applications to geometry and it is by no means
%	an obvious idea! If you're interested, look up the history of \emph{analytic geometry}, which is the formal name for what
%	we're doing.}. However, we need to take special care to note the assumptions we've made. After all, it would be
%	perfectly reasonable to claim that $S'=\Set{(y,x)\given y=4-x}$ (note the difference between $S$ and $S'$) 
%	is the solution set to Equation \eqref{SOLNSETEXAMPLE}.
%	However, it becomes less clear how $S'$ should be interpreted as a subset of $\R^2$. Should $(y,x)=(a,b)$ correspond to the
%	column vector $\mat{a\\b}$ or is $\mat{b\\a}$ the correct column vector? To solve this issue, we use the following convention.
%
%	\begin{emphbox}[Solution Set Convention] Unless otherwise specified,
%		solutions are converted to column vectors based on the order the variables are specified
%		in the original equations. This order will almost always be alphabetical if variables are assigned
%		different letters (e.g, $x$, $y$, $z$) or numerically if variables are indexed (e.g., $x_1$, $x_2$, $x_3$).
%	\end{emphbox}
%
%	Following this convention, $S$ is the ``correct'' solution set. That is, without specifying otherwise, the solutions
%	to $x+y=4$ correspond to the set $\Set*{\mat{a\\b}\given b=4-a}\subseteq \R^2$. This just so happens to agree with grade-school
%	convention\footnote{ It's as if there's a vast underworld of educators who plan how math should be taught from the time you
%	were five up until now.}.

	\Heading{Consistent \& Inconsistent Systems}

	Consider the following equations (as separate equations, not as a system):
	\[
		x^2=0\qquad\text{with solution set}\qquad S_x\subseteq \R,
	\]
	\[
		y^2=4\qquad\text{with solution set}\qquad S_y\subseteq \R,
	\]
	and
	\[
		z^2=-1\qquad\text{with solution set}\qquad S_z\subseteq \R.
	\]
	$S_x=\Set{0}$ consists of a single number. $S_y=\Set{2,-2}$ consists of two numbers, and $S_z=\Set{}$ consists of no
	numbers\footnote{ We're not allowing complex numbers at the moment.}.
	In this case, we would call the first two equations \emph{consistent} and the third equation \emph{inconsistent}.

	\begin{definition}[Consistent \& Inconsistent]
		An equation or system of equations is called \emph{consistent} if it has at least one solution. That
		is, its solution set is non-empty. Otherwise, an equation or system of equations is called \emph{inconsistent}.
	\end{definition}

	Why the word \emph{consistent}? This comes from the term \emph{logically consistent} which means ``able to be true''.
	An equation is an assertion that the left hand side equals the right hand side. If that can never happen, the assertion
	is not logically consistent.

	This terminology becomes more clear with systems. Consider the system
	\[
		\systeme{x-y=0,x-y=1}.
	\]
	From the first equation, we deduce $y=x$. From the second equation, we deduce $x=1+y$. Since $x=x$, we know that
	$y=x=1+y$ and therefore
	$y=1+y$. However, this is never true! There is a logical inconsistency between the two equations. In isolation they're
	fine, but taken together they're not.

	\Heading{Equivalent Systems}
	Two systems of equations are logically equivalent if they express the same relationships
	between their variables. For example, the equations $x=2y$ and $\tfrac{1}{2}x=y$ express the exact same
	relationship between the variables $x$ and $y$. This can be formalized using solution sets.

	\begin{definition}[Equivalent Systems]
		Two equations or systems of equations are called \emph{equivalent}\index[definitions]{Equivalent systems} if they have the same solution sets.
	\end{definition}

	Again, $x=2y$ and $\tfrac{1}{2}x=y$ both have the same solution set (a line through the origin of slope $\tfrac{1}{2}$),
	and so they are equivalent.

	Philosophical note: the process of ``doing algebra'' can be viewed as
	the process of \emph{manipulating equations/systems into easier to understand equivalent 
	equations/systems}. When you're asked to algebraically solve $x^2-4=0$. You might first factor to get the 
	equivalent equation $(x-2)(x+2)=0$. Then, since non-zero numbers cannot multiply to give zero, we know
	$x-2=0$ or $x+2=0$, which in turn is equivalent to $x=\pm2$. It's always been about equivalent systems!\footnote{
	Technically, up to this point we've been talking about \emph{conjunctive} systems, which means that a solution
	must hold for all equations of a system. The system $x=\pm 2$ is a \emph{disjunctive} system, which means a solution
	only needs to hold for \emph{one} of the equations ($x=2$ or $x=-2$), but the idea is the same.}

	\Heading{Row Reduction}

	Consider the vector equation
	\[
		t\vec u+s\vec v+r\vec w = \vec p\qquad\text{where}\qquad \vec u=\mat{1\\2\\1},\ 
		\vec v=\mat{2\\1\\-4},\ \vec w=\mat{-2\\-5\\1},\ \vec p=\mat{-15\\-21\\18}.
	\]
	By expanding in terms of coordinates, we get an equivalent system of linear equations.
	\begin{equation}
		\label{EQVECEQ2}
		\systeme[tsr]{
			t+2s-2r=-15@\qquad\text{row}_1,
			2t+s-5r=-21@\qquad\text{row}_2,
			t-4s+r=18@\qquad\text{row}_3
		}
	\end{equation}

	The most general way to solve any system is by \emph{substitution}. For System 
	\eqref{EQVECEQ2}, we could solve the first equation for $t$, substitute the result in
	the remaining equations, solve the next equation for $s$, etc.. However,
	because System \eqref{EQVECEQ2} is a \emph{linear} system, there's an alternate
	method: \emph{row reduction}\footnote{
	Row reduction is sometimes referred to as \emph{Gaussian elimination},
	\emph{Gauss-Jordan elimination}, or just \emph{elimination}; though there are subtle differences between Gaussian
	and Gauss-Jordan elimination,
	they aren't important, and we'll refer to all similar methods as \emph{row reduction}.}.

	Study the following manipulations of System \eqref{EQVECEQ2} and convince yourself
	that each operation produces a system equivalent to the one it came from.
	\begin{align}
	\sysdelim\{.
		\systeme[tsr]{
			t+2s-2r=-15,
			2t+s-5r=-21,
			t-4s+r=18
		}
		&\xrightarrow{\text{row}_3\mapsto\text{row}_3-\text{row}_1}
	\sysdelim\{.
		\systeme[tsr]{
			t+2s-2r=-15,
			2t+s-5r=-21,
			-6s+3r=33
		}\nonumber\\[4pt]
		&\xrightarrow{\text{row}_2\mapsto\text{row}_2-2\text{row}_1}
	\sysdelim\{.
		\systeme[tsr]{
			t+2s-2r=-15,
			-3s-r=9,
			-6s+3r=33
		}\nonumber\\[4pt]
		&\xrightarrow{\text{row}_3\mapsto\text{row}_3-2\text{row}_2}
	\sysdelim\{.
		\systeme[tsr]{
			t+2s-2r=-15,
			-3s-r=9,
			  5r=15
		}\label{EQEQUIVSYS}
	\end{align}
	From the final system, System \eqref{EQEQUIVSYS}, it's easy to see that $r=3$.
	From there,
	we can substitute $r=3$ into the second row of System \eqref{EQEQUIVSYS}
	to find $s=-4$ and we can substitute both $r$ and $s$ into the first row of System \eqref{EQEQUIVSYS}
	to find $t=-1$.

	By adding and subtracting rows, we ``reduced'' the number of variables from some equations until
	they were easy to solve. As an added benefit, every system along the way to System \eqref{EQEQUIVSYS}
	was nicely organized and formatted. In fact, the systems are so well organized that we can
	save time by not writing the variables and keeping track of the numbers in an \emph{augmented matrix}\footnote{
		A \emph{matrix} is a box of numbers. An \emph{augmented matrix} is a matrix with
		extra information associated with it.
	}. That is, instead of writing
	\[
		\systeme[tsr]{
			t+2s-2r=-15,
			2t+s-5r=-21,
			t-4s+r=18
		}
	\]
	we will write
	\[
		\begin{bmatrix}[rrr|r]
			1&2&-2 & -15\\
			2&1&-5&-21\\
			1&-4&1&18
		\end{bmatrix}.
	\]
	We call the matrix an \emph{augmented matrix} to stress that it contains two types of information:
	the \emph{coefficients} of the variables $t$, $s$, and $r$ and the \emph{constants} on
	the right hand side of the equations. An (optional) vertical line
	separates the two types of numbers.


	Now, the process of row reduction looks like this:
	\begin{align*}
		{\footnotesize
		\systeme[tsr]{
			t+2s-2r=-15,
			2t+s-5r=-21,
			t-4s+r=18
		}
		} \rightarrow
		\begin{bmatrix}[rrr|r]
			1&2&-2 & -15\\
			2&1&-5&-21\\
			1&-4&1&18
		\end{bmatrix}
		&\xrightarrow{\text{row}_3\mapsto\text{row}_3-\text{row}_1}
		\begin{bmatrix}[rrr|r]
			1&2&-2 & -15\\
			2&1&-5&-21\\
			0&-6&3&33
		\end{bmatrix}\\
		&\xrightarrow{\text{row}_2\mapsto\text{row}_2-2\text{row}_1}
		\begin{bmatrix}[rrr|r]
			1&2&-2 & -15\\
			0&-3&-1&9\\
			0&-6&3&33
		\end{bmatrix}\\
		&\xrightarrow{\text{row}_3\mapsto\text{row}_3-2\text{row}_2}
		\begin{bmatrix}[rrr|r]
			1&2&-2 & -15\\
			0&-3&-1&9\\
			0&0&5&15
		\end{bmatrix}
		\rightarrow
		{\footnotesize
			\systeme[tsr]{
			t+2s-2r=-15,
			-3s-r=9,
			  5r=15
			}
		}
	\end{align*}

	The operations are identical, but we write augmented matrices instead of equations.

	\begin{emphbox}[Takeaway]
		Augmented matrices are a notational tool that makes the process of doing row reduction more efficient.
	\end{emphbox}

	\Heading{The Rules of Row Reduction}

	So far, we've been able to row reduce systems by adding a multiple of one row to another\footnote{ Technically, we subtracted, but
	that's just adding a negative.}, but to fully solve \emph{any} system, we need additional operations\footnote{ If you're clever,
	you can think up alternatives to the elementary row operations that work just as well, but there's good reason to favor 
	the
	three elementary row operations. We'll see them when discussing matrix decompositions.}.
	\begin{definition}[Elementary Row Operations]
		The three \emph{elementary row operations}, which can be performed on a matrix or system of equations, are
		\smallskip
		\begin{itemize}
			\item swapping two rows (written $\text{row}_i\leftrightarrow \text{row}_j$),
			\item multiplying a row by a non-zero scalar (written $\text{row}_i\mapsto k\,\text{row}_i$), and
			\item adding a multiple of one row to another (written $\text{row}_i\mapsto \text{row}_i+ k\,\text{row}_j$).
		\end{itemize}
	\end{definition}

	Notice that each elementary row operation can be undone. For example, if you perform $\text{row}_i\mapsto k\,\text{row}_i$,
	you can undo it with $\text{row}_i\mapsto \tfrac{1}{k}\,\text{row}_i$. Therefore, applying an elementary row operation
	to a system is guaranteed to
	produce an equivalent system. 

	The strategy for solving a system is now summarized as:
	\begin{enumerate}
		\item Rewrite the system as an augmented matrix.
		\item Use elementary row operations to zero-out the lower triangle of the augmented matrix.
		\item Convert the matrix back to a system of equations.
		\item Read off the solution (substituting when necessary).
	\end{enumerate}

	\begin{example}
		Find a solution to the following system:
		\[
			\systeme{
				a+3b+2c=1,
				2a+7b+5c=2,
				-a-4b=11
			}.
		\]

		To do so, we rewrite the system as an augmented matrix then row reduce.
		\begin{align*}
			{\footnotesize
			\systeme{
				a+3b+2c=1,
				2a+7b+5c=2,
				-a-4b=11
			}
			} \rightarrow
			\begin{bmatrix}[rrr|r]
				1&3&2 & 1\\
				2&7&5 & 2\\
				-1&-4&0 & 11
			\end{bmatrix}
			&\xrightarrow{\text{row}_2\mapsto\text{row}_2-2\text{row}_1}
			\begin{bmatrix}[rrr|r]
				1&3&2 & 1\\
				0&1&1 & 0\\
				-1&-4&0 & 11
			\end{bmatrix}\\
			&\xrightarrow{\text{row}_3\mapsto\text{row}_3+\text{row}_1+\text{row}_2}
			\begin{bmatrix}[rrr|r]
				1&3&2 & 1\\
				0&1&1 & 0\\
				0&0&3 & 12
			\end{bmatrix}
			\rightarrow
			{\footnotesize
				\systeme{
				a+3b+2c=1,
				b+c=0,
				3c=12
				}
			}
		\end{align*}
		The third row reveals that $c=4$; substituting into the second row, we find $b=-4$. Now we can substitute $b=-4$ and $c=4$ into the first row and we obtain $a=5$.
		
		Thus, the solution is $\mat{a\\b\\c}=\mat{5\\-4\\4}$. Since this is the only solution to the system, the solution set is $\Set*{\mat{5\\-4\\4}}$.
	\end{example}

	\begin{example}
		Solve the system
		\[
			\systeme[tsr]{
				3t+s+13r=-2,
				t+5r=1,
				-t+s-6r=-8,
				t+s+4r=-6
			}.
		\]

		Again, we row reduce the corresponding augmented matrix to find an equivalent system\index{System of linear equations!equivalent} from which we can more easily compute the solution.
		\begin{align*}
			{\footnotesize
			\systeme[tsr]{
				3t+s+13r=-2,
				t+5r=1,
				-t+s-6r=-8,
				t+s+4r=-6
			}
			} \rightarrow
			\begin{bmatrix}[rrr|r]
				3&1&13 & -2\\
				1&0&5 & 1\\
				-1&1&-6 & -8\\
				1&1&4 & -6
			\end{bmatrix}
			&\xrightarrow{\text{row}_1\leftrightarrow\text{row}_2}
			\begin{bmatrix}[rrr|r]
				1&0&5 & 1\\
				3&1&13 & -2\\
				-1&1&-6 & -8\\
				1&1&4 & -6
			\end{bmatrix}\\
			&\xrightarrow{\text{row}_2\mapsto\text{row}_2-3\text{row}_1}
			\begin{bmatrix}[rrr|r]
				1&0&5 & 1\\
				0&1&-2 & -5\\
				-1&1&-6 & -8\\
				1&1&4 & -6
			\end{bmatrix}\\
			&\xrightarrow{\text{row}_3\mapsto\text{row}_3+\text{row}_1-\text{row}_2}
			\begin{bmatrix}[rrr|r]
				1&0&5 & 1\\
				0&1&-2 & -5\\
				0&0&1 & -2\\
				1&1&4 & -6
			\end{bmatrix}\\
			&\xrightarrow{\text{row}_4\mapsto\text{row}_4-\text{row}_1-\text{row}_2}
			\begin{bmatrix}[rrr|r]
				1&0&5 & 1\\
				0&1&-2 & -5\\
				0&0&1 & -2\\
				0&0&1 & -2
			\end{bmatrix}\\
			&\xrightarrow{\text{row}_4\mapsto\text{row}_4-\text{row}_3}
			\begin{bmatrix}[rrr|r]
				1&0&5 & 1\\
				0&1&-2 & -5\\
				0&0&1 & -2\\
				0&0&0 & 0
			\end{bmatrix}
			\rightarrow
			{\footnotesize
				\systeme[tsr]{
				t+5r=1,
				s-2r=-5,
				r=-2,
				0=0
				}
			}
		\end{align*}
		Our equivalent system reveals $r=-2$, which we can substitute back into the first and second rows to find that $t=11$ and $s=-9$.
		
		As a vector, the solution is $\mat{t\\s\\r}=\mat{11\\-9\\-2}$ and so the solution set is $\Set*{\mat{11\\-9\\-2}}$.
	\end{example}

	In the examples so far, we've stopped row reducing when the equations became simple enough
	to solve by inspection. However, we could continue row reducing until the system is as simple as possible.

	\begin{example}
		Solve the system
		\[
			\systeme{
				a+3b+2c=1,
				2a+7b+5c=2,
				-a-4b=11
			}
		\]

		Notice that we solved this system using a combination of row reduction and substitution in a previous example.
		This time, let us use only row reduction.
		
		The augmented matrix for this system is
		\[
			\begin{bmatrix}[rrr|r]
				1&3&2 & 1\\
				2&7&5 & 2\\
				-1&-4&0 & 11
			\end{bmatrix}.
		\]
		Based on the work from the previous example, we know it can be reduced to
		\[
			\begin{bmatrix}[rrr|r]
				1&3&2 & 1\\
				0&1&1 & 0\\
				0&0&3 & 12
			\end{bmatrix}.
		\]
		Now let us continue row reducing.
		\begin{align*}
			\begin{bmatrix}[rrr|r]
				1&3&2 & 1\\
				0&1&1 & 0\\
				0&0&3 & 12
			\end{bmatrix}
			&\xrightarrow{\text{row}_3\mapsto\tfrac{1}{3}\,\text{row}_3}
			\begin{bmatrix}[rrr|r]
				1&3&2 & 1\\
				0&1&1 & 0\\
				0&0&1 & 4
			\end{bmatrix}\\
			&\xrightarrow{\text{row}_2\mapsto\text{row}_2-\text{row}_3}
			\begin{bmatrix}[rrr|r]
				1&3&2 & 1\\
				0&1&0 & -4\\
				0&0&1 & 4
			\end{bmatrix}\\
			&\xrightarrow{\text{row}_1\mapsto\text{row}_1-3\text{row}_2-2\text{row}_3}
			\begin{bmatrix}[rrr|r]
				1&0&0 & 5\\
				0&1&0 & -4\\
				0&0&1 & 4
			\end{bmatrix}
			\rightarrow
			{\footnotesize
				\systeme{
				a=5,
				b=-4,
				c=4
				}
			}
		\end{align*}
		The solution is $\mat{a\\b\\c}=\mat{5\\-4\\4}$ and the solution set is $\Set*{\mat{5\\-4\\4}}$, which is the same as we got before.
	\end{example}

	What happens when you apply row reduction to an inconsistent system? Let's see. Consider the system
	\begin{equation}
		\label{EQSYSINCONEX}
		\systeme{x+y=1,4x+4y=7}.
	\end{equation}
	Before continuing, convince yourself that this system is inconsistent.
	The augmented matrix for System \eqref{EQSYSINCONEX} is
	\[
		\begin{bmatrix}[cc|c]1&1&1\\4&4&7\end{bmatrix}.
	\]
	We apply the row operation $\text{row}_2\mapsto \text{row}_2-4\text{row}_1$ to get
	\[
		\begin{bmatrix}[cc|c]1&1&1\\0&0&3\end{bmatrix},
	\]
	which corresponds to the system
	\[
		\systeme{x+y=1,0x+0y=3}.
	\]
	But, the last equation says $0x+0y=3$, which is not true for any choice of $x$ and $y$!
	Thus, we see applying row reduction to an inconsistent system reveals its inconsistency.

	\begin{example}
		Find a solution to the following system:
		\[
			\systeme{
				x+z=4,
				x+y+2z=-8,
				x+3y+4z=-18
			}.
		\]

		As usual we rewrite the system as an augmented matrix and then row reduce.
		\begin{align*}
			{\footnotesize
			\systeme{
				x+z=4,
				x+y+2z=-8,
				x+3y+4z=-18
			}
			} \rightarrow
			\begin{bmatrix}[rrr|r]
				1&0&1 & 4\\
				1&1&2 & -8\\
				1&3&4 & -18
			\end{bmatrix}
			&\xrightarrow{\text{row}_3\mapsto\text{row}_3-\text{row}_2}
			\begin{bmatrix}[rrr|r]
				1&0&1 & 4\\
				1&1&2 & -8\\
				0&2&2 & -10
			\end{bmatrix}\\
			&\xrightarrow{\text{row}_2\mapsto\text{row}_2-\text{row}_1}
			\begin{bmatrix}[rrr|r]
				1&0&1 & 4\\
				0&1&1 & -12\\
				0&2&2 & -10
			\end{bmatrix}\\
			&\xrightarrow{\text{row}_3\mapsto\text{row}_3-2\text{row}_2}
			\begin{bmatrix}[rrr|r]
				1&0&1 & 4\\
				0&1&1 & -12\\
				0&0&0 & 14
			\end{bmatrix}
			\rightarrow
			{\footnotesize
				\systeme{
				x+z=4,
				y+z=-12,
				0x+0y+0z=14
				}
			}
		\end{align*}
		The equation $0x + 0y + 0z = 14$ is never true and so the system is inconsistent.
		Since there are no values of $x$, $y$, and $z$ that satisfy the system, the solution set is $\Set*{}$, the empty set.
	\end{example}
